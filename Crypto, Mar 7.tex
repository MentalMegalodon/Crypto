% -*-latex-*-
\documentclass[12pt]{article}

\usepackage{amssymb}
\usepackage{amsmath}
\usepackage{amsthm}
\usepackage{cancel}

\usepackage[paper=letterpaper,includehead,left=1in,right=1in,bottom=1in,top=.6in,headheight=.6in]{geometry}
\usepackage{fancyhdr}
\fancyhead[L]{University of Texas at Austin\\
Department of Computer Science}
\fancyhead[R]{Cryptography\\
CS 346, Spring 2016}
\pagestyle{fancy}

\frenchspacing

\newcommand{\Z}{\mathbb{Z}}
\newcommand{\AAA}{\mathcal{A}}
\newcommand{\CCC}{\mathcal{C}}
\newcommand{\FFF}{\mathcal{F}}
\newcommand{\KKK}{\mathcal{K}}
\newcommand{\MMM}{\mathcal{M}}
\newcommand{\TTT}{\mathcal{T}}

\newcommand{\Concat}{\parallel}
\newcommand{\eps}{\varepsilon}
\newcommand{\Enc}{\mathsf{Enc}}
\newcommand{\Dec}{\mathsf{Dec}}
\newcommand{\Mac}{\mathsf{Mac}}
\newcommand{\Macf}{\mathsf{Mac\text{-}forge}}
\newcommand{\Macsf}{\mathsf{Mac\text{-}sforge}}
\newcommand{\Encf}{\mathsf{Enc\text{-}forge}}
\newcommand{\Vrfy}{\mathsf{Vrfy}}
\newcommand{\Decrypt}[2]{\Dec_{#1}(#2)}
\newcommand{\Encrypt}[2]{\Enc_{#1}(#2)}
\newcommand{\Gen}{\mathsf{Gen}}
\newcommand{\ang}[1]{\langle#1\rangle}
\newcommand{\GenEncDec}{(\Gen,\Enc,\Dec)}
\newcommand{\GenMacVrfy}{(\Gen,\Mac,\Vrfy)}
\newcommand{\ExptEavArgs}[2]{\mathsf{PrivK}^{\mathsf{eav}}_{#1,#2}}
\newcommand{\ExptCcaArgs}[2]{\mathsf{PrivK}^{\mathsf{CCA}}_{#1,#2}}
\newcommand{\ExptCpaArgs}[2]{\mathsf{PrivK}^{\mathsf{CPA}}_{#1,#2}}
\newcommand{\ExptHCArgs}[2]{\mathsf{Hash\text{-}coll}_{#1,#2}}
\newcommand{\ExptEav}{\ExptEavArgs{\AAA}{\Pi}}
\newcommand{\ExptCca}{\ExptCcaArgs{\AAA}{\Pi}}
\newcommand{\ExptCpa}{\ExptCpaArgs{\AAA}{\Pi}}
\newcommand{\ExptHC}{\ExptHCArgs{\AAA}{\Pi}}
\newcommand{\ExptPrgArgs}[2]{\mathsf{PRG}_{#1,#2}}
\newcommand{\ExptPrg}{\ExptPrgArgs{\AAA}{G}}
\newcommand{\Fcns}[1]{\mathsf{Func}_n}
\newcommand{\LengthKey}[1]{\ell_{\mathit{key}}(#1)}
\newcommand{\LengthInput}[1]{\ell_{\mathit{in}}(#1)}
\newcommand{\LengthOutput}[1]{\ell_{\mathit{out}}(#1)}
\newcommand{\xor}{\oplus}
\newcommand{\Pit}{\widetilde{\Pi}}
\newcommand{\negl}{{\tt negl}}

\newcommand{\Exec}[5]{\mathit{exec}^{#1}_{#2}(#3,#4,#5)}

\begin{document}

\title{CS 346 Class Notes}
\date{Mar 7, 2016}
\author{Mark Lindberg}
\maketitle
\thispagestyle{fancy}

{\bf Last Time:}

5.3 Message Authentication using Hash functions.

``Hash and Mac'' paradigm.

Construction 5.5:

$\Pi=(\Mac,\Vrfy)$. A fixed-length MAC for messages of length $\ell(n)$.

$\Pi_H=(\Gen_H,H)$. A hash function with output length $\ell(n)$.

Construct Mac: $\Pi'=(\Gen',\Mac',\Vrfy')$.

$\Gen'$: Run $\Gen_H$ to get $s$. Also get random $n$-bit $k$. Key is $(s,k)$.

$\Mac'_{s,k}(m)=\Mac_k(H^s(m))$.

$\Vrfy'_{s,k}(m)=\Vrfy_k(H^s(m),t)$.

{\bf This Time:}

Theorem 5.6: If $\Pi$ is secure and $\Pi_H$ is collision resistant, then $\Pi'$ is secure.

Proof: Let $\AAA'$ be an arbitrary PPT adversary in the $\Macf_{\AAA',\Pi'}(n)$ experiment.

Split the $\AAA'$ successes into ``Type I'' and ``Type II''.

$\AAA'$ succeeds if it produces $(m,t)$, such that $m\not\in Q$ (set of messages $\AAA'$ has previously made oracle calls on), and $\Vrfy'_{s,k}(m,t)=1$. From $\Vrfy'$ definition, $\Vrfy_k(H^s(m),t)=1$.

Note that $\Pr[\AAA'\text{ succeeds}] = \Pr[\AAA'\text{ has Type I success}] + \Pr[\AAA'\text{ has Type II success}]$.

\begin{itemize}

\item[Type I]: $H^s(m)=H^s(m')$ for some $m'\in Q$.

\item[Type II]: Otherwise.

\end{itemize}

Type I breaks collision resistance of $\Pi_H$. We can then relate Type II to breaking $\Pi$'s security.

Type I:

To prove: $\Pr[\AAA'\text{ has Type I success}]$ is $\negl$:

Construct a PPT adversary $\AAA_H$ in $\ExptHC(n)$ that simulates $\AAA'$.

$\AAA_H$ gets $s$ and outputs $m,m'$. It succeeds iff $m\neq m'$ and $H^s(m)=H^s(m')$.

To simulate $\AAA'$. $\AAA'$ calls oracle $\Mac'_{k,s}(m)=\Mac_k(H^s(m))$ at outset, $\AAA_H$ generates a random $n$-bit $k$ when $\AAA'$ outputs $(m,t)$.

If $\AAA'$ does not have a Type I success, $\AAA_H$ outputs arbitrary messages.

Otherwise, output $m,m'$ such that $H^s(m)=H^s(m')$ and $m'\in Q$.

Type II:

To prove: $\Pr[\AAA'\text{ has Type II success}]$ is $\negl$.

Construct a PPT adversary $\AAA$ for $\Macf_{A,\Pi}(n)$. $\AAA$ simulates $\AAA'$.

To simulate a call to $\underbrace{\Mac'_{s,k}(m)}_{\text{Oracle of }\AAA'}=\underbrace{\Mac_k(H^s(m))}_{\text{Oracle of }\AAA}$.

At outset, $\AAA$ run $\Gen_H(1^n)$ to get $s$. When $\AAA'$ outputs $m,t$, if $\AAA'$ does not get a Type II success, give an arbitrary output.

Otherwise, $\AAA$ outputs $\left(H^s(m),t\right)$. This will pass $\Vrfy$. Note that $H^s(m)\neq H^s(m')$ for any $m'\neq m$, $m,m'\in Q$, because it's a Type II success, and therefore cannot be a Type I success.

Done!

HMAC construction!

Um, I took a picture. That thing was ridiculous.

Generic birthday attacks on Hash functions.

THE BIRTHDAY PARADOX IS NOT A FREAKING PARADOX, DANGIT.

On a hash function with $\ell(n)$-bit output strings, $O(2^{\frac{\ell(n)}{2}})$ evaluations are sufficient to find a collision with good probability.

Analysis: Assume idealized Hash function (worst case), $n$ bins, throw balls into random bins until we have a collision. Expected time, $\Theta(\sqrt{n})$. Hehehe. This is mathematically provable, and makes perfect sense probabilistically, and there is no paradox. Period. *sigh*

Constant space birthday attack: Pick an IV, and keep hashing the output, and try to see when it loops on itself.

There's a solution to this. Huh.

\end{document}
























