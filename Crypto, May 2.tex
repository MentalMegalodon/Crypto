% -*-latex-*-
\documentclass[12pt]{article}

\usepackage{amssymb}
\usepackage{amsmath}
\usepackage{amsthm}
\usepackage{cancel}
\usepackage{hyperref}
\usepackage[utf8]{inputenc}
\usepackage[english]{babel}
\usepackage[usenames, dvipsnames]{color}
\usepackage{ mathrsfs }

\usepackage[paper=letterpaper,includehead,left=1in,right=1in,bottom=1in,top=.6in,headheight=.6in]{geometry}
\usepackage{fancyhdr}
\fancyhead[L]{University of Texas at Austin\\
Department of Computer Science}
\fancyhead[R]{Cryptography\\
CS 346, Spring 2016}
\pagestyle{fancy}

\frenchspacing

\newcommand{\Z}{\mathbb{Z}}
\newcommand{\N}{\mathbb{N}}
\newcommand{\R}{\mathbb{R}}
\newcommand{\G}{\mathbb{G}}
\newcommand{\AAA}{\mathcal{A}}
\newcommand{\CCC}{\mathcal{C}}
\newcommand{\FFF}{\mathcal{F}}
\newcommand{\KKK}{\mathcal{K}}
\newcommand{\MMM}{\mathcal{M}}
\newcommand{\TTT}{\mathcal{T}}

\newcommand{\Concat}{\parallel}
\newcommand{\eps}{\varepsilon}
\newcommand{\Enc}{\mathsf{Enc}}
\newcommand{\Dec}{\mathsf{Dec}}
\newcommand{\Encaps}{\mathsf{Encaps}}
\newcommand{\Decaps}{\mathsf{Decaps}}
\newcommand{\Sign}{\mathsf{Sign}}
\newcommand{\Mac}{\mathsf{Mac}}
\newcommand{\Macf}{\mathsf{Mac\text{-}forge}}
\newcommand{\Macsf}{\mathsf{Mac\text{-}sforge}}
\newcommand{\Encf}{\mathsf{Enc\text{-}forge}}
\newcommand{\Vrfy}{\mathsf{Vrfy}}
\newcommand{\Decrypt}[2]{\Dec_{#1}(#2)}
\newcommand{\Encrypt}[2]{\Enc_{#1}(#2)}
\newcommand{\Gen}{\mathsf{Gen}}
\newcommand{\GenRSA}{\mathsf{GenRSA}}
\newcommand{\GenM}{\mathsf{GenModulus}}
\newcommand{\ang}[1]{\langle#1\rangle}
\newcommand{\GenEncDec}{(\Gen,\Enc,\Dec)}
\newcommand{\GenMacVrfy}{(\Gen,\Mac,\Vrfy)}
\newcommand{\ExptEavArgs}[2]{\mathsf{PrivK}^{\mathsf{eav}}_{#1,#2}}
\newcommand{\ExptPubEavArgs}[2]{\mathsf{PubK}^{\mathsf{eav}}_{#1,#2}}
\newcommand{\ExptPubCpaArgs}[2]{\mathsf{PubK}^{\mathsf{LR-CPA}}_{#1,#2}}
\newcommand{\ExptPubCcaArgs}[2]{\mathsf{PubK}^{\mathsf{CCA}}_{#1,#2}}
\newcommand{\ExptCcaArgs}[2]{\mathsf{PrivK}^{\mathsf{CCA}}_{#1,#2}}
\newcommand{\ExptCpaArgs}[2]{\mathsf{PrivK}^{\mathsf{CPA}}_{#1,#2}}
\newcommand{\ExptKEMCPAArgs}[2]{\mathsf{KEM}^{\mathsf{CPA}}_{#1,#2}}
\newcommand{\ExptKEMCCAArgs}[2]{\mathsf{KEM}^{\mathsf{CCA}}_{#1,#2}}
\newcommand{\ExptHCArgs}[2]{\mathsf{Hash\text{-}coll}_{#1,#2}}
\newcommand{\ExptINVTArgs}[2]{\mathsf{Invert}_{#1,#2}}
\newcommand{\ExptRSAArgs}[2]{\mathsf{RSA\text{-}Inv}_{#1,#2}}
\newcommand{\ExptRSALSBArgs}[2]{\mathsf{RSA\text{-}LSB}_{#1,#2}}
\newcommand{\ExptDLogArgs}[2]{\mathsf{DLog}_{#1,#2}}
\newcommand{\ExptKEArgs}[3]{\mathsf{KE}_{#1,#2}^{#3}}
\newcommand{\FacArgs}[2]{\mathsf{Factor}_{#1,#2}}
\newcommand{\ExptEav}{\ExptEavArgs{\AAA}{\Pi}}
\newcommand{\ExptPubEav}{\ExptPubEavArgs{\AAA}{\Pi}}
\newcommand{\ExptPubCpa}{\ExptPubCpaArgs{\AAA}{\Pi}}
\newcommand{\ExptPubCca}{\ExptPubCcaArgs{\AAA}{\Pi}}
\newcommand{\ExptCca}{\ExptCcaArgs{\AAA}{\Pi}}
\newcommand{\ExptCpa}{\ExptCpaArgs{\AAA}{\Pi}}
\newcommand{\ExptKEMCPA}{\ExptKEMCPAArgs{\AAA}{\Pi}}
\newcommand{\ExptKEMCCA}{\ExptKEMCCAArgs{\AAA}{\Pi}}
\newcommand{\ExptHC}{\ExptHCArgs{\AAA}{\Pi}}
\newcommand{\ExptINVT}{\ExptINVTArgs{\AAA}{f}}
\newcommand{\ExptRSA}{\ExptRSAArgs{\AAA}{\GenRSA}}
\newcommand{\ExptRSALSB}{\ExptRSALSBArgs{\AAA}{\GenRSA}}
\newcommand{\Fac}{\ExptINVTArgs{\AAA}{\GenM}}
\newcommand{\ExptPrgArgs}[2]{\mathsf{PRG}_{#1,#2}}
\newcommand{\ExptPrg}{\ExptPrgArgs{\AAA}{G}}
\newcommand{\ExptKE}{\ExptKEArgs{\AAA}{\Pi}{EAV}}
\newcommand{\ExptDLog}{\ExptDLogArgs{\AAA}{G}}
\newcommand{\Fcns}[1]{\mathsf{Func}_n}
\newcommand{\LengthKey}[1]{\ell_{\mathit{key}}(#1)}
\newcommand{\LengthInput}[1]{\ell_{\mathit{in}}(#1)}
\newcommand{\LengthOutput}[1]{\ell_{\mathit{out}}(#1)}
\newcommand{\xor}{\oplus}
\newcommand{\Pit}{\widetilde{\Pi}}
\newcommand{\negl}{{\tt negl}}
\newcommand{\from}{\leftarrow}

\newcommand{\Exec}[5]{\mathit{exec}^{#1}_{#2}(#3,#4,#5)}

\begin{document}

\title{CS 346 Class Notes}
\date{May 2, 2016}
\author{Mark Lindberg}
\maketitle
\thispagestyle{fancy}

{\bf Last Time:}

LSB = Least Significant Bit.

Theorem 11.31. If RSA is hard relative to $\GenRSA$, then for all PPT $\AAA$, $$\Pr[\ExptRSALSB(n)=1]\leq\frac{1}{2}+\negl(n).$$

Saw how to compute the 2SB of $x$ when LSB$(x)=0$.

Given $N,e,y$, with $y=x^e$. Then let $z=\frac{x}{2}$. Then $y'=z^e=y\cdot2^{-e}$.

{\bf This Time:}

2 more days...

Recommended exercises.

Today: When LSB$(x)=1$... Formulas.

PKE for single-bit messages. (Without using the random oracle model.)

Construction 11.32.\begin{itemize}

\item $\Gen$: As in plain RSA. $\GenRSA\to(N,e,d)$, $pk=(N,e)$, $sk=(N,d)$.

\item $\Enc$: Input $pk,m\in\{0,1\}$. Pick a random $r\in\Z_N^*$ such that LSB$(r)=m$. Set $c=r^e$.

\item $\Dec$: Input $sk,c$. Get $r=c^d$. Output $m:=\text{LSB}(r)$.

\end{itemize}

Theorem $11.33$: If RSA problem is hard relative to $\GenRSA$, then construction 11.32 is CPA-secure. This follows easily from theorem 11.31.

Main result of this section:

CPA-secure KEM based on the RSA assumption, with no random oracle needed.

Construction 11.14.\begin{itemize}

\item $\Gen$: Run $\GenRSA(1^n)$ to get $(N,e,d)$ and $\phi(N)$. Compute $d'=d^n\mod{\phi(N)}$. Set $pk=(N,e)$, $sk=(N,d')$.

\item $\Encaps$: On input $pk$ and $1^n$, choose $c$, uniformly at random, from $\Z_N^*$. For $i=1$ to $n$\begin{enumerate}

\item Compute $k_i=\text{LSB}(c_i)$.

\item Set $c_{i+1}=c_i^e$

\end{enumerate}

Output ciphertext $c_{n+1}$, and $k=k_1\dots k_n$.

\item $\Decaps$: On input $sk=(N,d')$ and ciphertext $c$, let $c_1=c^{d'}$. For $i=1$ to $N$\begin{enumerate}

\item $k_i=\text{LSB}(c_i)$.

\item $c_{i+1}=c_i^e$.

\end{enumerate}

Output $k=k_1\dots k_n$.

Reason for correctness: $c_1=c_2^d=c_3^{d^2}=\dots=c_{n+1}^{d^n}=c_{n+1}^{d^n\mod{\phi(N)}}=c_{n+1}^{d'}$.

\end{itemize}

Theorem 11.35. If RSA is hard relative to $\GenRSA$, then construction 11.34 is CPA-secure.

Chapter 12: Digital Signature Schemes.\begin{itemize}

\item Public-key analogue of a MAC.

\item Provides integrity.

\item Provides other important properties such as\begin{itemize}

\item Public verifiability.

\item Transferability.

\item Non-repudiation.

\end{itemize}

\end{itemize}


12.2 Definitions.

$(\Gen,\Sign,\Vrfy)$. Key Generation, signing, verification.\begin{itemize}

\item $\Gen$: On input $1^n$ produces $(pk,sk)$ (Even number $\geq n$ bits). 

\item $\Sign$: On input \underline{$sk$}, and a message $m$, produces A signature $\sigma$. $\sigma\from\Sign_{sk}(m)$.

\item $\Vrfy$: On input $pk$, message $m$, and signature $\sigma$, $\Vrfy_{pk}(m,\sigma)$ outputs $1$ if it accepts the message, and $0$ else.

\end{itemize}

Definition of sigforge.

Definition 12.2

Section 12.3.

Theorem 12.4.

Section 12.4.

12.4.1. Plain RSA (NOT secure.)

\end{document}
























