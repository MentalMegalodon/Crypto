% -*-latex-*-
\documentclass[12pt]{article}

\usepackage{amssymb}
\usepackage{amsmath}
\usepackage{amsthm}
\usepackage{cancel}
\usepackage{hyperref}
\usepackage[utf8]{inputenc}
\usepackage[english]{babel}
\usepackage[usenames, dvipsnames]{color}

\usepackage[paper=letterpaper,includehead,left=1in,right=1in,bottom=1in,top=.6in,headheight=.6in]{geometry}
\usepackage{fancyhdr}
\fancyhead[L]{University of Texas at Austin\\
Department of Computer Science}
\fancyhead[R]{Cryptography\\
CS 346, Spring 2016}
\pagestyle{fancy}

\frenchspacing

\newcommand{\Z}{\mathbb{Z}}
\newcommand{\N}{\mathbb{N}}
\newcommand{\R}{\mathbb{R}}
\newcommand{\AAA}{\mathcal{A}}
\newcommand{\CCC}{\mathcal{C}}
\newcommand{\FFF}{\mathcal{F}}
\newcommand{\KKK}{\mathcal{K}}
\newcommand{\MMM}{\mathcal{M}}
\newcommand{\TTT}{\mathcal{T}}

\newcommand{\Concat}{\parallel}
\newcommand{\eps}{\varepsilon}
\newcommand{\Enc}{\mathsf{Enc}}
\newcommand{\Dec}{\mathsf{Dec}}
\newcommand{\Mac}{\mathsf{Mac}}
\newcommand{\Macf}{\mathsf{Mac\text{-}forge}}
\newcommand{\Macsf}{\mathsf{Mac\text{-}sforge}}
\newcommand{\Encf}{\mathsf{Enc\text{-}forge}}
\newcommand{\Vrfy}{\mathsf{Vrfy}}
\newcommand{\Decrypt}[2]{\Dec_{#1}(#2)}
\newcommand{\Encrypt}[2]{\Enc_{#1}(#2)}
\newcommand{\Gen}{\mathsf{Gen}}
\newcommand{\GenRSA}{\mathsf{GenRSA}}
\newcommand{\GenM}{\mathsf{GenModulus}}
\newcommand{\ang}[1]{\langle#1\rangle}
\newcommand{\GenEncDec}{(\Gen,\Enc,\Dec)}
\newcommand{\GenMacVrfy}{(\Gen,\Mac,\Vrfy)}
\newcommand{\ExptEavArgs}[2]{\mathsf{PrivK}^{\mathsf{eav}}_{#1,#2}}
\newcommand{\ExptCcaArgs}[2]{\mathsf{PrivK}^{\mathsf{CCA}}_{#1,#2}}
\newcommand{\ExptCpaArgs}[2]{\mathsf{PrivK}^{\mathsf{CPA}}_{#1,#2}}
\newcommand{\ExptHCArgs}[2]{\mathsf{Hash\text{-}coll}_{#1,#2}}
\newcommand{\ExptINVTArgs}[2]{\mathsf{Invert}_{#1,#2}}
\newcommand{\ExptRSAArgs}[2]{\mathsf{RSA-Inv}_{#1,#2}}
\newcommand{\FacArgs}[2]{\mathsf{Factor}_{#1,#2}}
\newcommand{\ExptEav}{\ExptEavArgs{\AAA}{\Pi}}
\newcommand{\ExptCca}{\ExptCcaArgs{\AAA}{\Pi}}
\newcommand{\ExptCpa}{\ExptCpaArgs{\AAA}{\Pi}}
\newcommand{\ExptHC}{\ExptHCArgs{\AAA}{\Pi}}
\newcommand{\ExptINVT}{\ExptINVTArgs{\AAA}{f}}
\newcommand{\ExptRSA}{\ExptINVTArgs{\AAA}{\GenRSA}}
\newcommand{\Fac}{\ExptINVTArgs{\AAA}{\GenM}}
\newcommand{\ExptPrgArgs}[2]{\mathsf{PRG}_{#1,#2}}
\newcommand{\ExptPrg}{\ExptPrgArgs{\AAA}{G}}
\newcommand{\Fcns}[1]{\mathsf{Func}_n}
\newcommand{\LengthKey}[1]{\ell_{\mathit{key}}(#1)}
\newcommand{\LengthInput}[1]{\ell_{\mathit{in}}(#1)}
\newcommand{\LengthOutput}[1]{\ell_{\mathit{out}}(#1)}
\newcommand{\xor}{\oplus}
\newcommand{\Pit}{\widetilde{\Pi}}
\newcommand{\negl}{{\tt negl}}

\newcommand{\Exec}[5]{\mathit{exec}^{#1}_{#2}(#3,#4,#5)}

\begin{document}

\title{CS 346 Class Notes}
\date{Apr 4, 2016}
\author{Mark Lindberg}
\maketitle
\thispagestyle{fancy}

{\bf This Time:}

Upcoming exam:

Study problem sets 3 and 4.

Authenticated encryption, up to and including Chinese Remainder Theorem.

The factoring assumption:

First attempt:

Pick a random $n$-bit number $N$. Adversary $\AAA$ gets $N$, outputs $x$. $\AAA$ wins if $x$ is a nontrivial factor of $N$. We'd like to say that $\AAA$ has $\leq\negl$ probability of success. This fails because, for example, $2$ is a factor of $50\%$ of numbers, and this gives way better than negligible success.

Actual factoring experiment:

Three parameters: $n$, $\GenM$.

$\GenM$ on input $1^n$, generates distinct $n$-bit primes $p,q$ and their product $N$, where $N$ is the modulus. (Note: $\GenM$ is allowed to fail with $\negl(n)$ probability.)

$\Fac(n)$: Run $\GenM(1^n)$ to get $N,p,q$. Gives $N,n$ to $\AAA$. $\AAA$ outputs $x$. $\AAA$ wins if $x=p$ or $x=q$.

The factoring assumption:

$\exists$ PPT $\GenM$ such that $\forall$ PPT $\AAA$, $\Pr[\Fac(n)=1]\leq\negl(n)$. It is believed that the above holds for ``basic'' $\GenM$.

RSA assumption: $\GenRSA(1^n)$ produces $N,e,d$. Run $\GenM(1^n)$, to get $N,p,q$. Determine an $e$ relatively prime to $\phi(N)=(p-1)(q-1)$. Determine $d$ as $e^{-1}\pmod{\phi(N)}$.

The experiment: $\ExptRSA(n)$: Run $\GenRSA$ to get $N$, $e$, $d$. Select a uniform random $y\in\Z_n^*$.

Give $\AAA$ $n,N,e,y$, $\AAA$ outputs $x$.

$\AAA$ succeeds if $x^e=y\pmod{N}$.

$f_e(x)=x^e$, $f_e$ corresponds to a permutation on $\Z_n^*$.

RSA Assumption:

$\exists \GenRSA$ such that $\forall$ PPT $\AAA$, $\Pr[\ExptRSA(n)=1]\leq\negl(n)$.

We do have, 8.2.5, that if factoring is polynomial time solvable, then the RSA problem is polynomial time solvable.

He then goes over ``textbook'' RSA. Sorry, but I completely zoned out because I've taught this subject before.

Cryptographic assumption in cyclic groups.

$G$ is a finite group. Let $g\in G$. $\ang{g}=\{g^0,g^1,\dots,g^{i-1}\}$, where $i$ is the least positive integer such that $g^i=1$. Note that $\forall g$, $g^{|G|}=1$.

$\ang{g}$ is the subgroup of $G$ generated by $g$, and it is a group of order $i$.

We say that $g$ has order $i$ in the group $G$.

Proposition 8.52: $g^x=g^{x\pmod{i}}$. Easy because $g^i=1$.

Proposition 8.53: $g^x=g^y\Rightarrow x\equiv y\pmod{i}$. Easy.

Proposition 8.54: $i\mid|G|$.

Proof is in theorem 8.14.

Corollary 8.55: If $|G|$ is a prime $p$, every element $g\in G$ except the identity element is a generator.

A generator is an element such that $\ang{g}=G$.

Caution: $\Z_N^*$ does not have prime order.

Theorem 8.56: For any prime $p$, $\Z_p^*$ is a cyclic group of order $p-1$.

Residues and subgroups. Again, I had an entire class on algebraic structures, so I'm having a hard time concentrating today.

\end{document}
























