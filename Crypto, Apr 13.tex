% -*-latex-*-
\documentclass[12pt]{article}

\usepackage{amssymb}
\usepackage{amsmath}
\usepackage{amsthm}
\usepackage{cancel}
\usepackage{hyperref}
\usepackage[utf8]{inputenc}
\usepackage[english]{babel}
\usepackage[usenames, dvipsnames]{color}
\usepackage{ mathrsfs }

\usepackage[paper=letterpaper,includehead,left=1in,right=1in,bottom=1in,top=.6in,headheight=.6in]{geometry}
\usepackage{fancyhdr}
\fancyhead[L]{University of Texas at Austin\\
Department of Computer Science}
\fancyhead[R]{Cryptography\\
CS 346, Spring 2016}
\pagestyle{fancy}

\frenchspacing

\newcommand{\Z}{\mathbb{Z}}
\newcommand{\N}{\mathbb{N}}
\newcommand{\R}{\mathbb{R}}
\newcommand{\G}{\mathbb{G}}
\newcommand{\AAA}{\mathcal{A}}
\newcommand{\CCC}{\mathcal{C}}
\newcommand{\FFF}{\mathcal{F}}
\newcommand{\KKK}{\mathcal{K}}
\newcommand{\MMM}{\mathcal{M}}
\newcommand{\TTT}{\mathcal{T}}

\newcommand{\Concat}{\parallel}
\newcommand{\eps}{\varepsilon}
\newcommand{\Enc}{\mathsf{Enc}}
\newcommand{\Dec}{\mathsf{Dec}}
\newcommand{\Mac}{\mathsf{Mac}}
\newcommand{\Macf}{\mathsf{Mac\text{-}forge}}
\newcommand{\Macsf}{\mathsf{Mac\text{-}sforge}}
\newcommand{\Encf}{\mathsf{Enc\text{-}forge}}
\newcommand{\Vrfy}{\mathsf{Vrfy}}
\newcommand{\Decrypt}[2]{\Dec_{#1}(#2)}
\newcommand{\Encrypt}[2]{\Enc_{#1}(#2)}
\newcommand{\Gen}{\mathsf{Gen}}
\newcommand{\GenRSA}{\mathsf{GenRSA}}
\newcommand{\GenM}{\mathsf{GenModulus}}
\newcommand{\ang}[1]{\langle#1\rangle}
\newcommand{\GenEncDec}{(\Gen,\Enc,\Dec)}
\newcommand{\GenMacVrfy}{(\Gen,\Mac,\Vrfy)}
\newcommand{\ExptEavArgs}[2]{\mathsf{PrivK}^{\mathsf{eav}}_{#1,#2}}
\newcommand{\ExptCcaArgs}[2]{\mathsf{PrivK}^{\mathsf{CCA}}_{#1,#2}}
\newcommand{\ExptCpaArgs}[2]{\mathsf{PrivK}^{\mathsf{CPA}}_{#1,#2}}
\newcommand{\ExptHCArgs}[2]{\mathsf{Hash\text{-}coll}_{#1,#2}}
\newcommand{\ExptINVTArgs}[2]{\mathsf{Invert}_{#1,#2}}
\newcommand{\ExptRSAArgs}[2]{\mathsf{RSA-Inv}_{#1,#2}}
\newcommand{\ExptDLogArgs}[2]{\mathsf{DLog}_{#1,#2}}
\newcommand{\FacArgs}[2]{\mathsf{Factor}_{#1,#2}}
\newcommand{\ExptEav}{\ExptEavArgs{\AAA}{\Pi}}
\newcommand{\ExptCca}{\ExptCcaArgs{\AAA}{\Pi}}
\newcommand{\ExptCpa}{\ExptCpaArgs{\AAA}{\Pi}}
\newcommand{\ExptHC}{\ExptHCArgs{\AAA}{\Pi}}
\newcommand{\ExptINVT}{\ExptINVTArgs{\AAA}{f}}
\newcommand{\ExptRSA}{\ExptINVTArgs{\AAA}{\GenRSA}}
\newcommand{\Fac}{\ExptINVTArgs{\AAA}{\GenM}}
\newcommand{\ExptPrgArgs}[2]{\mathsf{PRG}_{#1,#2}}
\newcommand{\ExptPrg}{\ExptPrgArgs{\AAA}{G}}
\newcommand{\ExptDLog}{\ExptDLogArgs{\AAA}{G}}
\newcommand{\Fcns}[1]{\mathsf{Func}_n}
\newcommand{\LengthKey}[1]{\ell_{\mathit{key}}(#1)}
\newcommand{\LengthInput}[1]{\ell_{\mathit{in}}(#1)}
\newcommand{\LengthOutput}[1]{\ell_{\mathit{out}}(#1)}
\newcommand{\xor}{\oplus}
\newcommand{\Pit}{\widetilde{\Pi}}
\newcommand{\negl}{{\tt negl}}

\newcommand{\Exec}[5]{\mathit{exec}^{#1}_{#2}(#3,#4,#5)}

\begin{document}

\title{CS 346 Class Notes}
\date{Apr 13, 2016}
\author{Mark Lindberg}
\maketitle
\thispagestyle{fancy}

{\bf This Time:}

Exam be done.

Prime order subgroup of $\Z_p^*$ ($p$ prime).

$p=rq+1$. $p,q$ prime. Order $q$ subgroup of $\Z_p^*$. $\{h^r\mod{p}\mid h\in\Z_p^*\}$. These are called the $r$th residuals modulo $p$.

It's very easy to show that this is a subgroup.

It is a bit more challenging to prove that the subgroup is order $q$.

Note that $p-1=rq$.

Idea of proof: Show that $f(h)=h^r\mod{p}$ is ``$r$-to-$1$''.

We need to show that groups of $r$ elements each map to 1 element in our new group.

A result from last time, which now becomes useful: Proposition 8.53. Let $G$ be a finite group, $g\in G$ an element of order $i$. Then $g^x=g^y\Leftrightarrow x\equiv y\pmod(i)$.

Corollary: If $g$ is a generator, $g^x=g^y\Leftrightarrow x\equiv y\pmod{|G|}$.

Back to the proof: Let $g$ be a generator of $\Z_p^*$. Therefore, $g$ has order $p-1$ by theorem 8.56. The $g^0,g^1,\dots,g^{p-2}$ are all of the elements of $\Z_p^*$.

Let $i,j\in\Z_{p-1}$.

Claim 1: $g^i$ and $g^j$ have the same $r$th residue mod $p$ if and only if $q\mid(i-j)$.

Proof: Corollary above implies that $g^{ri}\equiv g^{rj}$, working in $\Z_p^*$, if and only if $ri\equiv rj\pmod{p-1}$ if and only if $r(i-j)$ is a multiple of $p-1=qr$, if and only if $i-j$ is a multiple of $q$.

Claim 2: $(g^0)^r,(g^1)^r,\dots,(g^{q-1})^r$ are distinct.

Proof: Immediate from previous claim.

Claim 3: Let $\ell\in\Z_q$.

Then $g^{\ell}, g^{\ell+q}$, $g^{\ell+2q},\dots,g^{\ell+(r-1)q}$ all have the same $r$th residual modulo $p$.

Proof: Immediate from claim 1.

------------------------------------------

Advantages of this type of group $G$, the prime order subgroup:\begin{enumerate}

\item We can generate a uniform random element of $G$: Choose $h$ in $\Z_p^*$ uniformly at random. Take $h^r\mod{p}$.

\item We can identify a generator for $G$ efficiently: Repeat $1)$ until we get an element $\neq$ identity.

\item We can efficiently test whether $h\in\Z_p^*$ belongs to $G$. Claim: $h\in G\Leftrightarrow h^g=1$.

\end{enumerate}

Proof of $3)$: Let $h=g^i$ where $g$ is a generator of $\Z_p^*$. $i\in\Z_{p-1}$.

Claim 1: $h\in G\Leftrightarrow r\mid i$.

Proof: (IF) Assume $r\mid i$. Then $i=cr$, $h=g^{c^r}\Rightarrow h\in G$.

(ONLY IF) Assume $h\in G$. Then $h=(g^j)^r=g^{j^r}$ for some $j\in\Z_{p-1}$. By the corollary from the beginning of class, $i\equiv jr\pmod{p-1}$. Thus, $qr\mid(i-jr)\Rightarrow r\mid(i-jr)\Rightarrow r\mid i$.

Claim 2: $h^q=1\iff r\mid i$.

\begin{align*}
&h^q=1\\
\iff&g^{qi}=1=j^0\\
\iff&(p-1)\mid qi\\
\iff&rq\mid qi\\
\iff&r\mid i
\end{align*}

\underline{\bf ``Discrete Log'' Problem.}

New experiment. $\ExptDLog(n)$. We have $G$ as a group generation algorithm. It generates $(\G,q,g)$, where $\G$ is a cyclic group of order $q$, and $g$ is a generator of $\G$. Our security parameter is $||q||=n$, the number of bits in the binary representation of $q$.

$\ExptDLog(n)$:\begin{itemize}

\item Run $G(1^n)$ to get $(\G,q,g)$.

\item Pick $h$ uniformly at random from $G$.

\item $\AAA$ is given $G,q,g,h$, $\AAA$ outputs $x$.

\item $\AAA$ succeeds if and only if $g^x=h$.

\end{itemize}

Definition 8.6.7: Discrete log is hard relative to $G$ if $\forall$ PPT $\AAA$, $\Pr[\ExptDLog(n)=1]\leq\negl(n)$.

''Discrete log problem is hard'' $\exists G$...

Section 8.4.2: Construction collision resistant hash functions given that we assume the discrete log problem is hard.

Construction 8.78. That's a lot of numbers. There is an explanation of this in the text. (I tend to space out at the end of classes, sorry. 2 classes in a row does that to me.)

Claim: Construction 8.78 gives a collision resistant hash function assuming the discrete-log problem is hard relative to $G$.

Idea of proof: Show that a collision enables us to compute the discrete log of $h$.

\end{document}
























