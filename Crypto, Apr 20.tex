% -*-latex-*-
\documentclass[12pt]{article}

\usepackage{amssymb}
\usepackage{amsmath}
\usepackage{amsthm}
\usepackage{cancel}
\usepackage{hyperref}
\usepackage[utf8]{inputenc}
\usepackage[english]{babel}
\usepackage[usenames, dvipsnames]{color}
\usepackage{ mathrsfs }

\usepackage[paper=letterpaper,includehead,left=1in,right=1in,bottom=1in,top=.6in,headheight=.6in]{geometry}
\usepackage{fancyhdr}
\fancyhead[L]{University of Texas at Austin\\
Department of Computer Science}
\fancyhead[R]{Cryptography\\
CS 346, Spring 2016}
\pagestyle{fancy}

\frenchspacing

\newcommand{\Z}{\mathbb{Z}}
\newcommand{\N}{\mathbb{N}}
\newcommand{\R}{\mathbb{R}}
\newcommand{\G}{\mathbb{G}}
\newcommand{\AAA}{\mathcal{A}}
\newcommand{\CCC}{\mathcal{C}}
\newcommand{\FFF}{\mathcal{F}}
\newcommand{\KKK}{\mathcal{K}}
\newcommand{\MMM}{\mathcal{M}}
\newcommand{\TTT}{\mathcal{T}}

\newcommand{\Concat}{\parallel}
\newcommand{\eps}{\varepsilon}
\newcommand{\Enc}{\mathsf{Enc}}
\newcommand{\Dec}{\mathsf{Dec}}
\newcommand{\Mac}{\mathsf{Mac}}
\newcommand{\Macf}{\mathsf{Mac\text{-}forge}}
\newcommand{\Macsf}{\mathsf{Mac\text{-}sforge}}
\newcommand{\Encf}{\mathsf{Enc\text{-}forge}}
\newcommand{\Vrfy}{\mathsf{Vrfy}}
\newcommand{\Decrypt}[2]{\Dec_{#1}(#2)}
\newcommand{\Encrypt}[2]{\Enc_{#1}(#2)}
\newcommand{\Gen}{\mathsf{Gen}}
\newcommand{\GenRSA}{\mathsf{GenRSA}}
\newcommand{\GenM}{\mathsf{GenModulus}}
\newcommand{\ang}[1]{\langle#1\rangle}
\newcommand{\GenEncDec}{(\Gen,\Enc,\Dec)}
\newcommand{\GenMacVrfy}{(\Gen,\Mac,\Vrfy)}
\newcommand{\ExptEavArgs}[2]{\mathsf{PrivK}^{\mathsf{eav}}_{#1,#2}}
\newcommand{\ExptPubEavArgs}[2]{\mathsf{PubK}^{\mathsf{eav}}_{#1,#2}}
\newcommand{\ExptPubCpaArgs}[2]{\mathsf{PubK}^{\mathsf{LR-CPA}}_{#1,#2}}
\newcommand{\ExptPubCcaArgs}[2]{\mathsf{PubK}^{\mathsf{CCA}}_{#1,#2}}
\newcommand{\ExptCcaArgs}[2]{\mathsf{PrivK}^{\mathsf{CCA}}_{#1,#2}}
\newcommand{\ExptCpaArgs}[2]{\mathsf{PrivK}^{\mathsf{CPA}}_{#1,#2}}
\newcommand{\ExptHCArgs}[2]{\mathsf{Hash\text{-}coll}_{#1,#2}}
\newcommand{\ExptINVTArgs}[2]{\mathsf{Invert}_{#1,#2}}
\newcommand{\ExptRSAArgs}[2]{\mathsf{RSA-Inv}_{#1,#2}}
\newcommand{\ExptDLogArgs}[2]{\mathsf{DLog}_{#1,#2}}
\newcommand{\ExptKEArgs}[3]{\mathsf{KE}_{#1,#2}^{#3}}
\newcommand{\FacArgs}[2]{\mathsf{Factor}_{#1,#2}}
\newcommand{\ExptEav}{\ExptEavArgs{\AAA}{\Pi}}
\newcommand{\ExptPubEav}{\ExptPubEavArgs{\AAA}{\Pi}}
\newcommand{\ExptPubCpa}{\ExptPubCpaArgs{\AAA}{\Pi}}
\newcommand{\ExptPubCca}{\ExptPubCcaArgs{\AAA}{\Pi}}
\newcommand{\ExptCca}{\ExptCcaArgs{\AAA}{\Pi}}
\newcommand{\ExptCpa}{\ExptCpaArgs{\AAA}{\Pi}}
\newcommand{\ExptHC}{\ExptHCArgs{\AAA}{\Pi}}
\newcommand{\ExptINVT}{\ExptINVTArgs{\AAA}{f}}
\newcommand{\ExptRSA}{\ExptINVTArgs{\AAA}{\GenRSA}}
\newcommand{\Fac}{\ExptINVTArgs{\AAA}{\GenM}}
\newcommand{\ExptPrgArgs}[2]{\mathsf{PRG}_{#1,#2}}
\newcommand{\ExptPrg}{\ExptPrgArgs{\AAA}{G}}
\newcommand{\ExptKE}{\ExptKEArgs{\AAA}{\Pi}{EAV}}
\newcommand{\ExptDLog}{\ExptDLogArgs{\AAA}{G}}
\newcommand{\Fcns}[1]{\mathsf{Func}_n}
\newcommand{\LengthKey}[1]{\ell_{\mathit{key}}(#1)}
\newcommand{\LengthInput}[1]{\ell_{\mathit{in}}(#1)}
\newcommand{\LengthOutput}[1]{\ell_{\mathit{out}}(#1)}
\newcommand{\xor}{\oplus}
\newcommand{\Pit}{\widetilde{\Pi}}
\newcommand{\negl}{{\tt negl}}
\newcommand{\from}{\leftarrow}

\newcommand{\Exec}[5]{\mathit{exec}^{#1}_{#2}(#3,#4,#5)}

\begin{document}

\title{CS 346 Class Notes}
\date{Apr 18, 2016}
\author{Mark Lindberg}
\maketitle
\thispagestyle{fancy}

{\bf This Time:}

Chapter 11: Public-key encryption. PKE.

Definition 11-1. A PKE scheme is a set of PPT algorithms $\Gen\Enc\Dec$ where\begin{enumerate}

\item $\Gen(1^n)$ produces $(pk, sk)$. Assume $pk, sk$ are at least $n$ bits long, and can determine $n$ from $pk, sk$.

\item $\Enc_{pk}(m)$. Randomized, yields ciphertext $c$. $m$ is drawn from the message space, which can depend on $pk$.

\item $\Dec_{sk}(m)$. Computes $m$ or $\perp$, which indicates failure. Deterministic.

\end{enumerate}

Correctness: Except for the possibility with $\negl$ probability over $(pk,sk)$ produced by $\Gen$, $\Dec_{sk}(\Enc_{pk}(m))=m$ for all $m$ in message space.

PKE security notions. Start with indistinguishable in the presence of an eavesdropper. (EAV-security).

Experiment: $\ExptPubEav(n)$. $\Pi=\GenEncDec$.\begin{enumerate}

\item Run $\Gen(1^n)$ to get $(pk,sk)$.

\item $\AAA$ is given $pk$ (!) and outputs equal-length $m_0,m_1$. (From the message space associated with $pk$.

\item Choose a random bit $b$. $c\from\Enc_{pk}(m_b)$. Give $c$ to $\AAA$.

\item $\AAA$ outputs $b'$. $\AAA$ succeeds if $b'=b$.

\end{enumerate}

Definition 11.2: $\Pi$ is EAV-secure if $\forall$ PPT adversaries $\AAA$, $$\Pr[\ExptPubEav(n)=1]\leq\frac{1}{2}+\negl(n).$$

This experiment is analogous to the CPA experiment in the private key setting, since $\AAA$ can compute polynomially many encryptions, because it was given the public key $pk$. (It can make its own oracle.)

Proposition 11.3. If a PKE scheme satisfies definition 11.2, then it is CPA-secure.

Impossibility of perfectly secret PKE. Related to 11.1 (Also 11.2?)

Modify definition 11.2 to \begin{enumerate}

\item Allow $\AAA$ to take an arbitrary amount of time...

\item Require $\Pr[\ExptPubEav(n)=1]=\frac{1}{2}$.

\end{enumerate}

Proof of impossibility: Let $t$ denote the upper bound on random bits used by $\Enc_{pk}$ on input $m_0$.

$\AAA$ can \cancel{run} simulate $\Enc_{pk}(m_0)$ using all possible sequences of $t$ random bits, and see if $c$ is produced. Correctness is now guaranteed.

Insecurity of a deterministic PKE.

Theorem 11.4: No deterministic PKE scheme is secure:

It's easy to compute the $c$ and check with no random bits...

11.2.2: Multiple Encryptions.

Definition 11.5: A PKE scheme $\Pi$ has indistinguishable multiple encryptions if for all PPT $\AAA$, $$\Pr[\ExptPubCpa(n)=1]\leq\frac{1}{2}+\negl(n).$$

Theorem 11.6: (There is a lengthy proof of this in the textbook) If a PKE scheme is CPA-secure, then it is secure with respect to definition 11.5.

Encryption of arbitrary-length messages.

Claim 11.7:

CPA-secure PKE $\Pi$ for fixed length messages $\to$ CPA-secure $\Pi'$ for arbitrary length messages.

CCA-security: $\ExptPubCca(n)$. Main difference here is that $\AAA$ gets access to a decryption oracle, except it can't call $\Dec_{sk}(c)$.

Definition 11.8: $\Pi$ has indistinguishable encryption under a chosen-ciphertext attack if $\forall$ PPT $\AAA$, $$\Pr[\ExptPubCca(n)=1]\leq\frac{1}{2}+\negl(n).$$

The analogue of theorem 11.6 holds!

Unfortunately, claim 11.2 does not.

Section 11.3: Hybrid encryption and the KEM/DEM paradigm.

Private vs public key encryption:\begin{itemize}

\item PKE building blocks are {\it much} slower.

\item Also, PKE has greater ``expansion''.

\item No need for a shared key. ($O(n)$ vs $O(n^2)$).

\end{itemize}

KEM: Key-Encapsulation Mechanism. We'll use a PKE to implement this.

DEM: Data-Encapsulation Mechanism. We'll use private-key encryption here.

KEM=($\Gen$,Encaps, Decaps)\begin{itemize}

\item $\Gen(1^n)$ produces $(pk,sk)$.

\item Encaps$_{pk}(1^n)$ produces $(c,k)$, where length of $k$ = $\ell(n)$.

\item Decaps$_{sk}(c)$ produces $k$.

\end{itemize}

Figure 11.2 from the textbook. Whee. :P

\end{document}
























