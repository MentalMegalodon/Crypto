% -*-latex-*-
\documentclass[12pt]{article}

\usepackage{amssymb}
\usepackage{amsmath}
\usepackage{amsthm}
\usepackage{cancel}

\usepackage[paper=letterpaper,includehead,left=1in,right=1in,bottom=1in,top=.6in,headheight=.6in]{geometry}
\usepackage{fancyhdr}
\fancyhead[L]{University of Texas at Austin\\
Department of Computer Science}
\fancyhead[R]{Cryptography\\
CS 346, Spring 2016}
\pagestyle{fancy}

\frenchspacing

\newcommand{\Z}{\mathbb{Z}}
\newcommand{\AAA}{\mathcal{A}}
\newcommand{\CCC}{\mathcal{C}}
\newcommand{\FFF}{\mathcal{F}}
\newcommand{\KKK}{\mathcal{K}}
\newcommand{\MMM}{\mathcal{M}}
\newcommand{\TTT}{\mathcal{T}}

\newcommand{\Concat}{\parallel}
\newcommand{\eps}{\varepsilon}
\newcommand{\Enc}{\mathsf{Enc}}
\newcommand{\Dec}{\mathsf{Dec}}
\newcommand{\Mac}{\mathsf{Mac}}
\newcommand{\Macf}{\mathsf{Mac\text{-}forge}}
\newcommand{\Macsf}{\mathsf{Mac\text{-}sforge}}
\newcommand{\Encf}{\mathsf{Enc\text{-}forge}}
\newcommand{\Vrfy}{\mathsf{Vrfy}}
\newcommand{\Decrypt}[2]{\Dec_{#1}(#2)}
\newcommand{\Encrypt}[2]{\Enc_{#1}(#2)}
\newcommand{\Gen}{\mathsf{Gen}}
\newcommand{\ang}[1]{\langle#1\rangle}
\newcommand{\GenEncDec}{(\Gen,\Enc,\Dec)}
\newcommand{\GenMacVrfy}{(\Gen,\Mac,\Vrfy)}
\newcommand{\ExptEavArgs}[2]{\mathsf{PrivK}^{\mathsf{eav}}_{#1,#2}}
\newcommand{\ExptCcaArgs}[2]{\mathsf{PrivK}^{\mathsf{CCA}}_{#1,#2}}
\newcommand{\ExptCpaArgs}[2]{\mathsf{PrivK}^{\mathsf{CPA}}_{#1,#2}}
\newcommand{\ExptHCArgs}[2]{\mathsf{Hash\text{-}coll}_{#1,#2}}
\newcommand{\ExptEav}{\ExptEavArgs{\AAA}{\Pi}}
\newcommand{\ExptCca}{\ExptCcaArgs{\AAA}{\Pi}}
\newcommand{\ExptCpa}{\ExptCpaArgs{\AAA}{\Pi}}
\newcommand{\ExptHC}{\ExptHCArgs{\AAA}{\Pi}}
\newcommand{\ExptPrgArgs}[2]{\mathsf{PRG}_{#1,#2}}
\newcommand{\ExptPrg}{\ExptPrgArgs{\AAA}{G}}
\newcommand{\Fcns}[1]{\mathsf{Func}_n}
\newcommand{\LengthKey}[1]{\ell_{\mathit{key}}(#1)}
\newcommand{\LengthInput}[1]{\ell_{\mathit{in}}(#1)}
\newcommand{\LengthOutput}[1]{\ell_{\mathit{out}}(#1)}
\newcommand{\xor}{\oplus}
\newcommand{\Pit}{\widetilde{\Pi}}
\newcommand{\negl}{{\tt negl}}

\newcommand{\Exec}[5]{\mathit{exec}^{#1}_{#2}(#3,#4,#5)}

\begin{document}

\title{CS 346 Class Notes}
\date{Mar 2, 2016}
\author{Mark Lindberg}
\maketitle
\thispagestyle{fancy}

{\bf Last Time:}

A strongly universal function:

Let $p$ be prime.
 
 $\KKK=\Z_p\times\Z_p$
 
 $\MMM=\TTT=\Z_p$.
 
 Then $h_{a,b}(m)=(a\times m+b)\mod{p}$, where $(a,b)=k\in\KKK$.
 
To show the strong universality we need:

$\forall m,m',t,t'\in\Z_p$ such that $m\neq m'$,

$\Pr[h_{a,b}(m)=t\wedge h_{a,b}(m')=t']=\frac{1}{p^2}$,

where the probability is taken over the uniform choice of key $a,b$.

{\bf This Time:}

We'll prove it!

$(am+b)\mod{p}=t$, $(am'+b)\mod{p}=t'$.

We'll argue there is a unique key $a,b$ satisfying these equations.

Then, assume without loss of generality that $m'>m$. (Swap if necessary.) $t-t'\mod{p}=(a(m'-m))\mod{p}$.

Let $x\in\{1,\dots,p-1\}$.

Then $ix\mod{p}$, $jx\mod{p}$ differ for $i,j\in\Z_p$, $i\neq j$.

PBC: Assume $j>i$, without loss of generality, such that $jx\mod{p}=ix\mod{p}$. Then $x(j-i)\mod{p}=0$, then $p\mid x(j-i)$, since $p$ is prime, $p\mid x$ or $p\mid j-i$. But $x,j-i\in\{1,\dots,p-1\}$, so contradiction.

The above shows that since $m'-m\leq p$, then there $\exists!a$ satisfying the equation. Then we can solve for $b$, and show that there is a unique key. Since there is a unique key, we see that probability is indeed $\frac{1}{p^2}$.

\underline{ \bf Chapter 5. Hash functions and applications.}

Definition of a hash function: $(\Gen, H)$

$\Gen(1^n)$ runs in polynomial time, and returns a key $s$. (We assume $n$ is implicit in $s$.)

For any binary string $x$, $H^s(kx)$ is an $\ell(n)$-bit binary string.

Fixed-length version: If $H^s$ is only defined for strings of length $\ell'(n)$, where $\ell'(n)>\ell(n)$, it is called a fixed-length hash function for inputs of length $\ell'(n)$. ``compression function''

\underline{Collision resistance}: It is hard to find input string hashing to the same output strings.

$\ExptHC(n)$. Run $\Gen(1^n)\to s$. $\AAA$ is given $s$, and also $n$, and outputs $x$, $x'$. The experiment outputs $1$ iff $x\neq x'$ and $H^s(x)=H^s(x')$.

In the fixed-length case, we also require that $|x|=|x'|=\ell'(n)$.

$(\Gen,H)$ is collision resistant if $\forall$ PPT adversary $\AAA$, $\Pr[\ExptHC(n)=1]\leq\negl(n)$.

Merkle-Damg\r{a}rd Transform:

Shows how to use a collision resistant fixed-length hash function $(\Gen,h)$ to obtain a collision resistant hash function $(\Gen,H)$ for arbitrary-length strings.

We'll assume that $\ell(n)=n$, and $\ell'(n)=2n$ for $h$.

Construction: Let $IV=0^n$. $x=x_1,x_2,\dots,x_d$, where $|x_1|=|x_2|=\dots=|x_d|=\ell(n)$, where last block is padded if necessary. Note that $d=\left\lceil\frac{|x|}{m}\right\rceil$. THIS IS WRONG: FIX WHEN UNDERSTAND $s$: Then $z_1=h^{0^n}(x_1)$, $z_2=h^{h_1}(x_2)$, $\dots$. We introduce another block $m_{d+1}$ which is the $n$-bit binary encoding of $|x|$. (Requires $|x|<2^n$, ridiculously easy if $n=128$ or something.)

Claim: $\Pi=(\Gen,h)$ is collision resistant $\Rightarrow$ $\Pi'=(\Gen,H)$ is collision resistant.

Let $\AAA'$ be an arbitrary adversary in the experiment $\ExptHCArgs{\AAA'}{\Pi'}(n)$. We'll construct an $\AAA$ for $\Pi$ from $\AAA'$. How do we do this?

Run $\Gen(1^n)$, as in $\AAA'$, to get $s$. Simulate $\AAA'$ to get $x,x'$. Problem: $\AAA$ needs to output $2n$-bit strings. We have 2 cases:\begin{enumerate}

\item $|x|\neq|x'|$. Then we can feed in the last blocks and have a collision in the original.

\item $|x|=|x'|$. Walk from the back until we find the first non-equal block, and that one gives a collision in the original.

(Yeah, I don't understand this fully. I'll revise these notes after I've read the textbook on this...)

\end{enumerate}

Example: Given $\Pi_1=(\Gen_1,H_1)$, and $\Pi_2=(\Gen_2,H_2)$, and we know that (at least) one of these is collision resistant. We'll define a $\Pi(\Gen,H)$ where $\Gen$ runs $\Gen_1\to s_1$, $\Gen_2\to s_2$, and output $s_1,s_2$. Then $H{s_1,s_2}(x)=H_1^{s_1}(x)||H_2^{s_2}(x)$. $\Pi$ is collision resistant:

Proof-ish: If we had an $\AAA$ for $\Pi$, we could define $\AAA_1$ and $\AAA_2$ for $\Pi_1$ and $\Pi_2$ respectively, and where we have a collision for $x,x'$ with $\Pi$, we have collisions for $x,x'$ with $\Pi_1$ AND $\Pi_2$ with $s_1$ and $s_2$ respectively. Then neither are secure. :)

\end{document}
























