% -*-latex-*-
\documentclass[12pt]{article}

\usepackage{amssymb}
\usepackage{amsmath}
\usepackage{amsthm}
\usepackage{cancel}
\usepackage{hyperref}
\usepackage[utf8]{inputenc}
\usepackage[english]{babel}
\usepackage[usenames, dvipsnames]{color}

\usepackage[paper=letterpaper,includehead,left=1in,right=1in,bottom=1in,top=.6in,headheight=.6in]{geometry}
\usepackage{fancyhdr}
\fancyhead[L]{University of Texas at Austin\\
Department of Computer Science}
\fancyhead[R]{Cryptography\\
CS 346, Spring 2016}
\pagestyle{fancy}

\frenchspacing

\newcommand{\Z}{\mathbb{Z}}
\newcommand{\AAA}{\mathcal{A}}
\newcommand{\CCC}{\mathcal{C}}
\newcommand{\FFF}{\mathcal{F}}
\newcommand{\KKK}{\mathcal{K}}
\newcommand{\MMM}{\mathcal{M}}
\newcommand{\TTT}{\mathcal{T}}

\newcommand{\Concat}{\parallel}
\newcommand{\eps}{\varepsilon}
\newcommand{\Enc}{\mathsf{Enc}}
\newcommand{\Dec}{\mathsf{Dec}}
\newcommand{\Mac}{\mathsf{Mac}}
\newcommand{\Macf}{\mathsf{Mac\text{-}forge}}
\newcommand{\Macsf}{\mathsf{Mac\text{-}sforge}}
\newcommand{\Encf}{\mathsf{Enc\text{-}forge}}
\newcommand{\Vrfy}{\mathsf{Vrfy}}
\newcommand{\Decrypt}[2]{\Dec_{#1}(#2)}
\newcommand{\Encrypt}[2]{\Enc_{#1}(#2)}
\newcommand{\Gen}{\mathsf{Gen}}
\newcommand{\ang}[1]{\langle#1\rangle}
\newcommand{\GenEncDec}{(\Gen,\Enc,\Dec)}
\newcommand{\GenMacVrfy}{(\Gen,\Mac,\Vrfy)}
\newcommand{\ExptEavArgs}[2]{\mathsf{PrivK}^{\mathsf{eav}}_{#1,#2}}
\newcommand{\ExptCcaArgs}[2]{\mathsf{PrivK}^{\mathsf{CCA}}_{#1,#2}}
\newcommand{\ExptCpaArgs}[2]{\mathsf{PrivK}^{\mathsf{CPA}}_{#1,#2}}
\newcommand{\ExptHCArgs}[2]{\mathsf{Hash\text{-}coll}_{#1,#2}}
\newcommand{\ExptEav}{\ExptEavArgs{\AAA}{\Pi}}
\newcommand{\ExptCca}{\ExptCcaArgs{\AAA}{\Pi}}
\newcommand{\ExptCpa}{\ExptCpaArgs{\AAA}{\Pi}}
\newcommand{\ExptHC}{\ExptHCArgs{\AAA}{\Pi}}
\newcommand{\ExptPrgArgs}[2]{\mathsf{PRG}_{#1,#2}}
\newcommand{\ExptPrg}{\ExptPrgArgs{\AAA}{G}}
\newcommand{\Fcns}[1]{\mathsf{Func}_n}
\newcommand{\LengthKey}[1]{\ell_{\mathit{key}}(#1)}
\newcommand{\LengthInput}[1]{\ell_{\mathit{in}}(#1)}
\newcommand{\LengthOutput}[1]{\ell_{\mathit{out}}(#1)}
\newcommand{\xor}{\oplus}
\newcommand{\Pit}{\widetilde{\Pi}}
\newcommand{\negl}{{\tt negl}}

\newcommand{\Exec}[5]{\mathit{exec}^{#1}_{#2}(#3,#4,#5)}

\begin{document}

\title{CS 346 Class Notes}
\date{Mar 21, 2016}
\author{Mark Lindberg}
\maketitle
\thispagestyle{fancy}

{\bf Last Time:}

HW stuff.

4.29:

$$\begin{bmatrix}
k_{1,1}&k_{1,2}&k_{1,3}&\cdots&{1,n}\\
k_{2,1}&k_{1,1}&k_{1,2}&\cdots&{2,n}\\
\vdots&\vdots&\vdots&\ddots&\vdots\\
k_{\ell,1}&k_{\ell-1,2}&k_{\ell-2,3}&\cdots&{\ell-n+1,1}\\
\end{bmatrix}$$

Somehow it's diagonal filled, so defining the first column and row define the entire matrix.

$k_{\ell-n+1,1}=k_{\max(1,\ell-n+1),max(1,n-\ell+1)}$.

{\bf This Time:}

Whirlwind tour of chapter 6.

\underline{Practical constructions of symmetric key primitives.}

6.1. \underline{Stream Ciphers.}

These are analogous to PRGs.

6.1.1 \underline{Linear Feedback Shift Register}

There are $n$ bits of state. For each bit of output, shift each of the bits of the state, $s_0$ shifts off as our next random bit, and $s_{n-1}$ will be replaced with the $\xor$ of some subset of the remaining bits. This can be implemented extremely efficiently in hardware.

Seeing $2n$ output bits is enough to determine the initial state $s_0,s_n-1$ {\it and} the subset of bits which are $\xor$ed together to form each successive $s_{n-1}$ value.

This is not a good proxy for a PRG at all.

6.1.2 \underline{Adding Nonlinearity}

\begin{enumerate}

\item Nonlinear feedback function.

\item Output bit is a non-linear function of the state.

\item ``Nonlinear combination generators''.

\end{enumerate}

Trinium: Developed in 2008.

Based on 3 LFSRs.

93 bit LFSR A, 84 bit LFSR B, and 111 bit LFSR C, for a total of 288 bits.

There's a really complicated diagram of how this actually works.

1152 pre-computer iterations is the magic number. :P

Older: RC4 - No longer recommended for use.

Designed for fast software implementation.

It uses byte operations and array indexing.

It is initialized with $(S,i,j)$, where $S$ is a 256-byte array, and $i$ and $j$ are indices in the array.

The key is $16$ bytes (in our example).

Again, there is a weird diagram for how it is initialized.

\href{https://en.wikipedia.org/wiki/RC4}{\textcolor{blue}{\underline{Wikipedia explanation.}}}

\ 

6.2 \underline{Block Ciphers}. (Practical implementation of strong PRPs.)

Fixed key length, and input length = output length (block length).

DES: Block length is 64 bits. Key length is 56 bits.

Triple DES: Key length is 112 bits.

Substitution-Permutation Networks:

Based on Shannon's ``confusion-diffusion paradigm''

Block length: 128 bits.

There are $2^{128}!$ permutations. HOLY. That number has $10^{40}$ DIGITS. If we index from $0$, we need $\log_2(2^{128}!)$ bits, which is $>2^128$. Yeah.

There's another diagram.

Substitution Permutation Networks: SPN.

Use fixed permutations for the f's. Where is the key used?\begin{enumerate}

\item Key mixing - xor input with a round subkey.

\item Substitution - confusion via fixed permutations.

\item Permutation - Diffusion.

\end{enumerate}

\end{document}
























