% -*-latex-*-
\documentclass[12pt]{article}

\usepackage{amssymb}
\usepackage{amsmath}
\usepackage{amsthm}
\usepackage{cancel}

\usepackage[paper=letterpaper,includehead,left=1in,right=1in,bottom=1in,top=.6in,headheight=.6in]{geometry}
\usepackage{fancyhdr}
\fancyhead[L]{University of Texas at Austin\\
Department of Computer Science}
\fancyhead[R]{Cryptography\\
CS 346, Spring 2016}
\pagestyle{fancy}

\frenchspacing

\newcommand{\AAA}{\mathcal{A}}
\newcommand{\CCC}{\mathcal{C}}
\newcommand{\FFF}{\mathcal{F}}
\newcommand{\KKK}{\mathcal{K}}
\newcommand{\MMM}{\mathcal{M}}

\newcommand{\Concat}{\parallel}
\newcommand{\eps}{\varepsilon}
\newcommand{\Enc}{\mathsf{Enc}}
\newcommand{\Dec}{\mathsf{Dec}}
\newcommand{\Decrypt}[2]{\Dec_{#1}(#2)}
\newcommand{\Encrypt}[2]{\Enc_{#1}(#2)}
\newcommand{\Gen}{\mathsf{Gen}}
\newcommand{\GenEncDec}{(\Gen,\Enc,\Dec)}
\newcommand{\ExptEavArgs}[2]{\mathsf{PrivK}^{\mathsf{eav}}_{#1,#2}}
\newcommand{\ExptEav}{\ExptEavArgs{\AAA}{\Pi}}
\newcommand{\ExptPrgArgs}[2]{\mathsf{PRG}_{#1,#2}}
\newcommand{\ExptPrg}{\ExptPrgArgs{\AAA}{G}}
\newcommand{\Fcns}[1]{\mathsf{Func}_n}
\newcommand{\LengthKey}[1]{\ell_{\mathit{key}}(#1)}
\newcommand{\LengthInput}[1]{\ell_{\mathit{in}}(#1)}
\newcommand{\LengthOutput}[1]{\ell_{\mathit{out}}(#1)}
\newcommand{\xor}{\oplus}
\newcommand{\Pit}{\widetilde{\Pi}}

\newcommand{\Exec}[5]{\mathit{exec}^{#1}_{#2}(#3,#4,#5)}

\begin{document}

\title{CS 346 Class Notes}
\date{Feb 8, 2016}
\author{Mark Lindberg}
\maketitle
\thispagestyle{fancy}

{\bf Review:}

Let $F$ be a length-preserving, efficient, keyed function. $F$ is a PRF if for all PPT distinguishers $D$, $$|\Pr(D^{F_k}(1^m)=1)-\Pr(D^{f}(1^m)=1)|$$ is negligible where $k$ is a random $m$-bit string and $f$ is a random function from $m$-bit strings to $m$-bit strings. A keyed function has $\ell_{in}(m)$, $\ell_{out}(m)$, $\ell_{key}(m)$. In a length-preserving function, all of these polynomials are equal to $m$.

{\bf Today:}

Pseudorandom Permutation (PRP).\begin{itemize}

\item $\ell_{in}(m)=\ell_{out}(m)$, and ``$F_k$" is a permutation.

\item Following the text, we will again assume that it is length-preserving for simplification.

\item If we say that a PRP is ``efficient'', we mean that both $F_k$ and $F_{k}^{-1}$ can be computed in polynomial time.

\end{itemize}

In the definition of PRP, ``$f$'' of the definition of a PRF is now a random permutation. There are $(2^{n})!$ possible permutations that $f$ is being drawn from.

Proposition 3.27: If $F$ is a PRP, then it is a PRF.

Proof idea: No PPT algorithm can tell the difference between a random function and a random permutation.

$m^3$ steps. To find a collision with good probability, need to make (tilde)$\approx\sqrt{2^m}=2^{\frac{m}{2}}$ oracle calls. Birthday problem. It is not a freaking paradox.

Strong PRP, $D$ gets two oracles, $D^{F_k,F_k^{-1}}$. $D^{f,f^{-1}}$.

Connection to practice: ``block ciphers'' are PRPs, generally for a specific key length.

Last time, we saw our first CPA secure scheme, which was based on a PRF. We encrypted by doing $\Encrypt{k}{m}=(r,m\xor F_k(r))$.

CPA security scheme for messages of $>>n$ bit lengths. (A polynomial message length.)

Today we'll see how to handle an arbitrary polynomial message length.

Relevant section of the textbook: Block cipher modes of operation (3.6.2).\begin{enumerate}

\item ECB mode: Electronic CodeBook. We get blocks, $m_1,m_2,\dots$ of the message, each of which is $n$ bits long. We take and output $F_k(m_1),F_k(m_2),\dots$. This is weak because, for example, if $m_1=m_5$, $c_1=F_k(m_1)=F_k(m_5)=c_5$. Repeated blocks encrypt to the same value. This fails EAV security trivially, and so is not even remotely CPA secure.

\item CBC mode: Cipher Block Chaining. We have an initial random value $IV$, $c_0=IV$, $c_1=F_k(m_1\xor IV)$, $c_2=F_k(m_2\xor c_1)$, $c_3=F_k(m_3\xor c_2)$, and so on.\begin{itemize}

\item Known to be CPA secure.

\item Minor variations on this scheme are insecure. Example: Cannot simply replace $IV$ with a counter. If you know what the $IV$ will be, an adversary can choose a $m_1$ that gives the same $\xor$, and therefore, the same $c_1$, that was produced last time.

Let $x=0^{n-1}1$, oracle call gives back $(IV,c)$, repeat if $IV$ is odd, so get back one where $IV$ is even. Use $m_0=0^m$, $m_1\neq m_0$ (anything else). Challenge ciphertext looks like $(IV+1, c')$. Notice that if $b=0$, then $c=c'$ (Assuming $IV$ is even, $IV\xor1=IV+1$.). If $b=1$, $c\neq c'$.

\item Another insecure variant: ``chained'' CBC. If you reuse the last ciphertext, $c_n$, as the $IV$ for the next message, you fails. Suppose I know $m_1$ is either $m_1^0$ or $m_1^1$. Assume a 3-block message, next message is $m_4,m_5$. Set $m_4=IV\xor m_1^0\xor c_3$. Then $c_4=F_k(m_4\xor c_3)=F_k(IV\xor m_1^0)$, which was the first input to give $c_1=F_k(IV\xor m_1^0)$, or $m_1^1$. This allows us to tell which value it was, and so we have broken CPA security.

\end{itemize}

\item OFB mode. Output FeedBack. We have $c_0=IV$, $c_1=m_1\xor F_k(IV)$, $c_2=m_2\xor F_k^2(IV)$, $c_3=m_3\xor F_k^3(IV)$, where $F_k^n$ means $F_k$ applied $n$ times.\begin{itemize}

\item In this mode we get an ``unsynchronized'' stream cipher. In a synchronized stream cipher, communicators generate a long stream, and use up parts of it with each message. Here, we have a key $k$ and an $IV$, and the $IV$ changes each time.

\end{itemize}

\item CTR mode, CounTeR. $c_0=CTR$, $c_1=m_1\xor F_k(CTR+1)$, $c_2=m_2\xor F_k(CTR+2)$. \begin{itemize}

\item Fully parallelizable. If there are multiple CPUs available, you can compute $c_n$ without knowing $c_{n-1}$, and so they can each encrypt their own blocks.

\item CTR mode is CPA-secure. Similar to theorem 3.31 (maybe 3.32?) given last time. If $m=m_1$, then $c=(r,F_k(r)\xor m)$. Given last time, let $\Pi$ be this encryption scheme.\begin{itemize}

\item $\Pit$ cannot be distinct from $\Pi$.

\item Show $\Pit$ is CPA-secure. Run in $q(m)$ time. Assume each message passed to the encryption oracle $\leq q(m)$ in length. The probability of overlap with counters used is $\leq\frac{2q(m)^2}{2^n}$, which is negligible.

\end{itemize}

\end{itemize}

\end{enumerate}

\end{document}




















