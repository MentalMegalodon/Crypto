% -*-latex-*-
\documentclass[12pt]{article}

\usepackage{amssymb}
\usepackage{amsmath}
\usepackage{amsthm}
\usepackage{cancel}
\usepackage{hyperref}
\usepackage[utf8]{inputenc}
\usepackage[english]{babel}
\usepackage[usenames, dvipsnames]{color}

\usepackage[paper=letterpaper,includehead,left=1in,right=1in,bottom=1in,top=.6in,headheight=.6in]{geometry}
\usepackage{fancyhdr}
\fancyhead[L]{University of Texas at Austin\\
Department of Computer Science}
\fancyhead[R]{Cryptography\\
CS 346, Spring 2016}
\pagestyle{fancy}

\frenchspacing

\newcommand{\Z}{\mathbb{Z}}
\newcommand{\AAA}{\mathcal{A}}
\newcommand{\CCC}{\mathcal{C}}
\newcommand{\FFF}{\mathcal{F}}
\newcommand{\KKK}{\mathcal{K}}
\newcommand{\MMM}{\mathcal{M}}
\newcommand{\TTT}{\mathcal{T}}

\newcommand{\Concat}{\parallel}
\newcommand{\eps}{\varepsilon}
\newcommand{\Enc}{\mathsf{Enc}}
\newcommand{\Dec}{\mathsf{Dec}}
\newcommand{\Mac}{\mathsf{Mac}}
\newcommand{\Macf}{\mathsf{Mac\text{-}forge}}
\newcommand{\Macsf}{\mathsf{Mac\text{-}sforge}}
\newcommand{\Encf}{\mathsf{Enc\text{-}forge}}
\newcommand{\Vrfy}{\mathsf{Vrfy}}
\newcommand{\Decrypt}[2]{\Dec_{#1}(#2)}
\newcommand{\Encrypt}[2]{\Enc_{#1}(#2)}
\newcommand{\Gen}{\mathsf{Gen}}
\newcommand{\ang}[1]{\langle#1\rangle}
\newcommand{\GenEncDec}{(\Gen,\Enc,\Dec)}
\newcommand{\GenMacVrfy}{(\Gen,\Mac,\Vrfy)}
\newcommand{\ExptEavArgs}[2]{\mathsf{PrivK}^{\mathsf{eav}}_{#1,#2}}
\newcommand{\ExptCcaArgs}[2]{\mathsf{PrivK}^{\mathsf{CCA}}_{#1,#2}}
\newcommand{\ExptCpaArgs}[2]{\mathsf{PrivK}^{\mathsf{CPA}}_{#1,#2}}
\newcommand{\ExptHCArgs}[2]{\mathsf{Hash\text{-}coll}_{#1,#2}}
\newcommand{\ExptEav}{\ExptEavArgs{\AAA}{\Pi}}
\newcommand{\ExptCca}{\ExptCcaArgs{\AAA}{\Pi}}
\newcommand{\ExptCpa}{\ExptCpaArgs{\AAA}{\Pi}}
\newcommand{\ExptHC}{\ExptHCArgs{\AAA}{\Pi}}
\newcommand{\ExptPrgArgs}[2]{\mathsf{PRG}_{#1,#2}}
\newcommand{\ExptPrg}{\ExptPrgArgs{\AAA}{G}}
\newcommand{\Fcns}[1]{\mathsf{Func}_n}
\newcommand{\LengthKey}[1]{\ell_{\mathit{key}}(#1)}
\newcommand{\LengthInput}[1]{\ell_{\mathit{in}}(#1)}
\newcommand{\LengthOutput}[1]{\ell_{\mathit{out}}(#1)}
\newcommand{\xor}{\oplus}
\newcommand{\Pit}{\widetilde{\Pi}}
\newcommand{\negl}{{\tt negl}}

\newcommand{\Exec}[5]{\mathit{exec}^{#1}_{#2}(#3,#4,#5)}

\begin{document}

\title{CS 346 Class Notes}
\date{Mar 23, 2016}
\author{Mark Lindberg}
\maketitle
\thispagestyle{fancy}

{\bf Last Time:}

Chapter 6: \underline{Practical constructions}.

{\bf This Time:}

Continued!

Substitution-Permutation Networks: (SPNs). (AES block cipher is an example of this.)

Each ``round'' has 3 OPS:\begin{enumerate}

\item Key mixing. $x\to x\xor k$, where $k$ is the sub-key for the round $i$.

\item Substitution. Break up $x$ into pieces and apply ``s-boxes''. (Each s-box sounds like a permutation.)

\item Permutation. Permute the bits of $x$ post-substitution by step 2.

\end{enumerate}

Given the key, $k$ (And hence, $\forall i$, $k_i$), then each action of each operation, and therefore the whole process, is invertible. Permutations can be inverted, because they are one to one. With the sub-key, we can ``undo'' the key mixing as well.

Remark: The last round of all 3 OPS should be followed by an additional key mixing step, else the substitution and permutation steps are completely reversible, and therefore pointless.

Feistel Networks: (We'll focus on the ``balanced'' case.)

We take our $n$-bit input and divide it into two halves. (Balance means we can assume clean division.)

Call these halves $L_0$ and $R_0$, both of which are $\frac{n}{2}$-bits.

Let $f_i$ be the round function for round $i$.

For round $i$, compute $R_i=L_{i-1}\xor f_i(R_{i-1})$, and $L_i=R_{i-1}$. (Run the computation and swap them.)

Here, we {\it can} let $f_i$ be non-invertible.

To invert the whole thing, $R_{i-1}=L_i$, and $L_{i-1}=R_i\xor f_{i}(R_{i-1})$.

DES uses a 16-round balanced Feistel Network.\begin{itemize}

\item The round function is the same in each round (Except for round sub-key). The function is sometimes called the DES ``mangler'' function. It has an SPN-like structure.

\end{itemize}

In DES, the block side is 64 bits. The key size is 56 bits. (This is short, one can brute-force it today because of advances in computing.) [There is a variation called triple-DES which addresses this.]

Our round function takes a $64/2=32$-bit input (And produces a 32-bit output, of course).

The sub-key in each round is 48-bits, formed by a permutation of a subset of the key's bites.

There is a trivial expansion step during which half of the bits of the input are duplicated to get a 48-bit result.

This is XORed with the sub key. This gives the key-mixing step.

Next, We break the mixed key into 8x 6-bit chunks in the obvious way.

The $i$th chunk goes to s-box $s_i$ which gives us a ``permutation'' which produces 4 bits. This is the analogue of step 2. It's a little different in that the value shrinks. We get a 32-bit value from the s-boxes.

There is then a 32-bit permutation which is applied to the 32 bits, giving us a 32-bit output.

Mangler function!

Increasing the key-size of a block cipher. (Using DES as an example.)

Idea 1: ``double encryption'', $F_{k_2}(F_{k_1}(x))=y$, but this still admits an attack using roughly $2^{56}$ evaluations.

Idea 2: ``triple encryption'', $F_{k_1}(F_{k_2}(F_{k_1}(x)))=y$. This one needs $2^{112}$ evaluations, so it's good and secure.

\underline{Hash functions in practice!}

SHA, SHA-2, and others, are based on a Davies-Meyer construction (Function $:2n\to n$ plus the Merkle-Damg\r{a}rd function.

Davies-Meyer\begin{itemize}

\item Uses a block cipher to get a compression function.

\end{itemize}

Can be justified in ``ideal-cipher'' model. That is, it can be proved collision resistant in the ``ideal-cipher'' model.

SHA-3.

We don't need to bound the running time of $\AAA$. Just need to bound the number of block cipher evaluations.

\end{document}
























