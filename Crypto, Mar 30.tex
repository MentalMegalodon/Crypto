% -*-latex-*-
\documentclass[12pt]{article}

\usepackage{amssymb}
\usepackage{amsmath}
\usepackage{amsthm}
\usepackage{cancel}
\usepackage{hyperref}
\usepackage[utf8]{inputenc}
\usepackage[english]{babel}
\usepackage[usenames, dvipsnames]{color}

\usepackage[paper=letterpaper,includehead,left=1in,right=1in,bottom=1in,top=.6in,headheight=.6in]{geometry}
\usepackage{fancyhdr}
\fancyhead[L]{University of Texas at Austin\\
Department of Computer Science}
\fancyhead[R]{Cryptography\\
CS 346, Spring 2016}
\pagestyle{fancy}

\frenchspacing

\newcommand{\Z}{\mathbb{Z}}
\newcommand{\N}{\mathbb{N}}
\newcommand{\R}{\mathbb{R}}
\newcommand{\AAA}{\mathcal{A}}
\newcommand{\CCC}{\mathcal{C}}
\newcommand{\FFF}{\mathcal{F}}
\newcommand{\KKK}{\mathcal{K}}
\newcommand{\MMM}{\mathcal{M}}
\newcommand{\TTT}{\mathcal{T}}

\newcommand{\Concat}{\parallel}
\newcommand{\eps}{\varepsilon}
\newcommand{\Enc}{\mathsf{Enc}}
\newcommand{\Dec}{\mathsf{Dec}}
\newcommand{\Mac}{\mathsf{Mac}}
\newcommand{\Macf}{\mathsf{Mac\text{-}forge}}
\newcommand{\Macsf}{\mathsf{Mac\text{-}sforge}}
\newcommand{\Encf}{\mathsf{Enc\text{-}forge}}
\newcommand{\Vrfy}{\mathsf{Vrfy}}
\newcommand{\Decrypt}[2]{\Dec_{#1}(#2)}
\newcommand{\Encrypt}[2]{\Enc_{#1}(#2)}
\newcommand{\Gen}{\mathsf{Gen}}
\newcommand{\ang}[1]{\langle#1\rangle}
\newcommand{\GenEncDec}{(\Gen,\Enc,\Dec)}
\newcommand{\GenMacVrfy}{(\Gen,\Mac,\Vrfy)}
\newcommand{\ExptEavArgs}[2]{\mathsf{PrivK}^{\mathsf{eav}}_{#1,#2}}
\newcommand{\ExptCcaArgs}[2]{\mathsf{PrivK}^{\mathsf{CCA}}_{#1,#2}}
\newcommand{\ExptCpaArgs}[2]{\mathsf{PrivK}^{\mathsf{CPA}}_{#1,#2}}
\newcommand{\ExptHCArgs}[2]{\mathsf{Hash\text{-}coll}_{#1,#2}}
\newcommand{\ExptINVTArgs}[2]{\mathsf{Invert}_{#1,#2}}
\newcommand{\ExptEav}{\ExptEavArgs{\AAA}{\Pi}}
\newcommand{\ExptCca}{\ExptCcaArgs{\AAA}{\Pi}}
\newcommand{\ExptCpa}{\ExptCpaArgs{\AAA}{\Pi}}
\newcommand{\ExptHC}{\ExptHCArgs{\AAA}{\Pi}}
\newcommand{\ExptINVT}{\ExptINVTArgs{\AAA}{f}}
\newcommand{\ExptPrgArgs}[2]{\mathsf{PRG}_{#1,#2}}
\newcommand{\ExptPrg}{\ExptPrgArgs{\AAA}{G}}
\newcommand{\Fcns}[1]{\mathsf{Func}_n}
\newcommand{\LengthKey}[1]{\ell_{\mathit{key}}(#1)}
\newcommand{\LengthInput}[1]{\ell_{\mathit{in}}(#1)}
\newcommand{\LengthOutput}[1]{\ell_{\mathit{out}}(#1)}
\newcommand{\xor}{\oplus}
\newcommand{\Pit}{\widetilde{\Pi}}
\newcommand{\negl}{{\tt negl}}

\newcommand{\Exec}[5]{\mathit{exec}^{#1}_{#2}(#3,#4,#5)}

\begin{document}

\title{CS 346 Class Notes}
\date{Mar 23, 2016}
\author{Mark Lindberg}
\maketitle
\thispagestyle{fancy}

{\bf This Time:}

Chapter 8.1. Preliminaries and basic group theory.

Proposition 8.2. If $a,b\in\N$, $\exists X,Y\in\Z$ s.t. $X\cdot a+Y\cdot b=\gcd(a,b)$, and $\gcd(a,b)$ is the least positive integer that can be written in this way.

Proof: Let $I=\{x\in\Z|\exists X^*,Y^*\in\Z\ x=aX^*+bY^*\}$. Note that $a,b\in I$. Let $d$ denote the minimum positive integer in $I$. Let $X'$, $Y'$ be integers such that $d=aX'+bY'$. Note: $\forall c\in I$, $d\mid c$. (And therefore, $d|a$ and $d|b$.) Note that there are $X'',Y''\in\Z$ such that $c=aX''+bY''$. Then we can write $c=qd+r$, where $0\leq r < d$, and $q,r\in\Z$. Therefore, $r=aX''+bY''-q(aX'+bY')=a(X''-qX')+b(Y''-qY')$. Therefore, $r\in I$. We noted that $d$ was the minimum positive integer in $I$, and since $0\leq r < d$, this means that $r=0$, and therefore, $c=qd$, and $d\mid c$.

Also $\neg\exists d'>d$ such that $d'\mid a$ and $d'\mid b$. Suppose that there was such a $d'>d$ such that $d'|a$ and $d'|b$. But then, $d'\mid a\cdot X'$ and $d'\mid b\cdot Y'$. Then $d'\mid(aX'+bY')$, but $aX'+bY'=d$, and this contradicts the fact that $d'>d$. Therefore, $d=\gcd(a,b)$.

Extended Euclidean Algorithm! A polynomial time algorithm to compute the $\gcd(a,b)$ as above.

Proposition 8.74. $b,N\in\N$, $b\geq 1$, $N>1$. $b$ is ``invertible'' modulo $N$ iff $\gcd(b,N)=1$.

Invertible: $\exists c$ such that $bc\equiv1\pmod{N}$.

$(\Leftarrow)$: Assume $\exists c$. Then $bc=1+\gamma N$, for some integer $\gamma$. Then $bc-\gamma N=1$, so by proposition 8.2, $\gcd(b,N)=1$.

$(\Rightarrow)$: Assume $\gcd(b,N)=1$. Then $\exists X,Y$ such that $bX+NY=1$ by proposition 8.2. Then $bX=1-NY$, and so $bX\equiv1\pmod{N}$. Therefore, $X$ is a multiplicative inverse of $b$ modulo $N$.

Groups: A set of elements $G$ and a binary operator $\circ:G\times G\to G$ such that\begin{enumerate}

\item Identity: $\exists e\in G$ such that $\forall e\in G$, $x\circ g=g\circ e=g$. This element must be unique.

\item Inverse: $\forall g\in G$, $\exists h\in G$ such that $g\circ h=e$.

\item Associative: $(g\circ g')\circ g''=g\circ(g'\circ g'')$.

\end{enumerate}

The order of a finite group $G$, written $|G|$, is the number of elements in $G$.

In an abelian group, we have commutativity.

There are a bunch of examples. I kinda spaced out because I've taken a few group theory math classes.

Theorem 8.14. If $G$ is a finite abelian group, and $g\in G$, then $g^{|G|}=1$.

Corollary 8.15. $g^x=g^{(x\mod{m})}$, where $m=|G|$.

Corollary 8.17. Define $f_i:G\to G$ as $f_i(g)=g^i$.

Let $e$ be such that $\gcd(e,m)='$.

Led $d$ be $e^{-1}\mod{m}$, so that $de\equiv1\pmod{m}$.

Then $f_e$, $f_d$ are permutations and $f_d$ is the inverse permutation of $f_e$.

$\Z_N$ is $\Z_N^+$, the set of all $\{0,1,\dots,N-1\}$ with addition.

$\Z_N^*$ is $\{i|1\leq i<N\wedge\gcd(i,N)=1\}$. All the invertible numbers mod $N$. Then $\Z_15^*=\{1,2,4,7,8,11,13,14\}$.

\end{document}
























