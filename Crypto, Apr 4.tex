% -*-latex-*-
\documentclass[12pt]{article}

\usepackage{amssymb}
\usepackage{amsmath}
\usepackage{amsthm}
\usepackage{cancel}
\usepackage{hyperref}
\usepackage[utf8]{inputenc}
\usepackage[english]{babel}
\usepackage[usenames, dvipsnames]{color}

\usepackage[paper=letterpaper,includehead,left=1in,right=1in,bottom=1in,top=.6in,headheight=.6in]{geometry}
\usepackage{fancyhdr}
\fancyhead[L]{University of Texas at Austin\\
Department of Computer Science}
\fancyhead[R]{Cryptography\\
CS 346, Spring 2016}
\pagestyle{fancy}

\frenchspacing

\newcommand{\Z}{\mathbb{Z}}
\newcommand{\N}{\mathbb{N}}
\newcommand{\R}{\mathbb{R}}
\newcommand{\AAA}{\mathcal{A}}
\newcommand{\CCC}{\mathcal{C}}
\newcommand{\FFF}{\mathcal{F}}
\newcommand{\KKK}{\mathcal{K}}
\newcommand{\MMM}{\mathcal{M}}
\newcommand{\TTT}{\mathcal{T}}

\newcommand{\Concat}{\parallel}
\newcommand{\eps}{\varepsilon}
\newcommand{\Enc}{\mathsf{Enc}}
\newcommand{\Dec}{\mathsf{Dec}}
\newcommand{\Mac}{\mathsf{Mac}}
\newcommand{\Macf}{\mathsf{Mac\text{-}forge}}
\newcommand{\Macsf}{\mathsf{Mac\text{-}sforge}}
\newcommand{\Encf}{\mathsf{Enc\text{-}forge}}
\newcommand{\Vrfy}{\mathsf{Vrfy}}
\newcommand{\Decrypt}[2]{\Dec_{#1}(#2)}
\newcommand{\Encrypt}[2]{\Enc_{#1}(#2)}
\newcommand{\Gen}{\mathsf{Gen}}
\newcommand{\ang}[1]{\langle#1\rangle}
\newcommand{\GenEncDec}{(\Gen,\Enc,\Dec)}
\newcommand{\GenMacVrfy}{(\Gen,\Mac,\Vrfy)}
\newcommand{\ExptEavArgs}[2]{\mathsf{PrivK}^{\mathsf{eav}}_{#1,#2}}
\newcommand{\ExptCcaArgs}[2]{\mathsf{PrivK}^{\mathsf{CCA}}_{#1,#2}}
\newcommand{\ExptCpaArgs}[2]{\mathsf{PrivK}^{\mathsf{CPA}}_{#1,#2}}
\newcommand{\ExptHCArgs}[2]{\mathsf{Hash\text{-}coll}_{#1,#2}}
\newcommand{\ExptINVTArgs}[2]{\mathsf{Invert}_{#1,#2}}
\newcommand{\ExptEav}{\ExptEavArgs{\AAA}{\Pi}}
\newcommand{\ExptCca}{\ExptCcaArgs{\AAA}{\Pi}}
\newcommand{\ExptCpa}{\ExptCpaArgs{\AAA}{\Pi}}
\newcommand{\ExptHC}{\ExptHCArgs{\AAA}{\Pi}}
\newcommand{\ExptINVT}{\ExptINVTArgs{\AAA}{f}}
\newcommand{\ExptPrgArgs}[2]{\mathsf{PRG}_{#1,#2}}
\newcommand{\ExptPrg}{\ExptPrgArgs{\AAA}{G}}
\newcommand{\Fcns}[1]{\mathsf{Func}_n}
\newcommand{\LengthKey}[1]{\ell_{\mathit{key}}(#1)}
\newcommand{\LengthInput}[1]{\ell_{\mathit{in}}(#1)}
\newcommand{\LengthOutput}[1]{\ell_{\mathit{out}}(#1)}
\newcommand{\xor}{\oplus}
\newcommand{\Pit}{\widetilde{\Pi}}
\newcommand{\negl}{{\tt negl}}

\newcommand{\Exec}[5]{\mathit{exec}^{#1}_{#2}(#3,#4,#5)}

\begin{document}

\title{CS 346 Class Notes}
\date{Apr 4, 2016}
\author{Mark Lindberg}
\maketitle
\thispagestyle{fancy}

{\bf This Time:}

Corollary 8.21 to Theorem 8.14. For any $N>1$, and $a\in\Z_N^*$, $a^{\phi(N)}=1\pmod{N}$.

Corollary 8.22 (to corollary 8.17): Fix $N>1$. For any integer $i$, define $f_i:\Z_N^*\to\Z_N^*$ as $f_i(x)=x^i$. Let $e$ be such that $\gcd(e,\phi(N))=1$, and let $d=e^{-1}\pmod{\phi(N)}$. Then $f_d$ and $f_e$ are permutations and $f_d$ is the inverse of $f_e$.

The elements of $\Z_p^*$ are $1,2,\dots,p-1$ for $p$ prime. If $N=p\cdot q$ for distinct primes $p,q$, then $|Z_N^*|=(p-1)(q-1)$.

Define: $\phi(N)=|\Z_N^*|$ for $N>1$.

Modular exponentiation: $x^{1001101_2}=x^{10000000_2}\cdot x^{1000_2}\cdot x^{100_2}\cdot x^{1_2}$.

$x^{1_2}=x$.

$x^{10_2}=(x^{1_2})^2$.

$x^{100_2}=(x^{10_2})^2$.

And so on.

Chinese Remainder Theorem!

\underline{Group Isomorphism}:

Let $G$ and $H$ be groups such that $|G|=|H|$.

Let the function $f:G\to H$ be a bijection, such that $\forall g_1,g_2\in G$, $f(g_1\circ_Gg_2)=f(g_1)\circ_Hf(g_2)$.

It is an easy exercise to show that $f^{-1}$ is a bijection from $H$ to $G$.

We say that if such an $f$ exists, $G$ and $H$ are isomorphic, $G\simeq H$.

\underline{CHINESE REMAINDER THEOREM}:

Let $p,q$ be distinct primes, $N=p\cdot q$. Then $\Z_N\simeq \Z_p\times \Z_q$, and $\Z_N^*\simeq \Z_p^*\times\Z_q^*$.

The direct product $G\times H$ of two groups $G,H$.

The elements of $G\times H$ are $($elements of $G)\times($elements of $H)$.

The operation in $G\times H$: $(g_1,h_1)\circ(g_2,h_2):=(g_1\circ_Gg_2,h_1\circ_Hh_2)$.

In the CRT, the isomorphism $f$ from $\Z_n$ to $\Z_p\times \Z_q$ is $f(x):=(x\pmod p, x\pmod q)$. The restriction of $f$ to $\Z_n^*$ is an isomorphism from $\Z_N^*$ to $\Z_p^*\times \Z_p^*$.

\underline{Applications of the CRT}:

$14\times13\pmod{15}$. Note that $N=15$, and $14,13\in\Z_{15}^*$. Let $p=5$, $q=3$.

$14\to(14\pmod 5, 14\pmod 3)=(4,2)$.

$13\to(13\pmod 5, 14\pmod3)=(3,1)$.

$(4,2)\times(3,1)=(4\times 3\pmod 5,1\times 2\pmod3)=(2,2)$.

Now, find a $y\in\Z_15^*$ such that $y\pmod5=2$, and $y\pmod3=2$. Obviously, $y=2$. (CRT guarantees exactly 1 element that satisfies this condition.)

For a fast way to do the other direction, when it's not so obvious, we can use the Extended Euclidean Algorithm to map $(x_p,x_q)$ back to $x\in\Z_N^*$.\begin{enumerate}

\item Solve the problem for $x_p=1$, $x_q=0$ to get $1_p\in\Z_N$.

Solve the problem for $x_p=0$, $x_q=1$ to get $1_q\in\Z_N$.

Find the $X,Y$ such that $pX+qY=1$. Then $1_p=X$, and $1_q=Y$.

\item $(x_p,x_q)\to x_p\cdot 1_p+x_q\cdot 1_q\pmod{N}$.

\end{enumerate}

Can we factor in polynomial time?

The naive algorithm (testing up to the square root) is exponential in size of the number.

There is no known polynomial time algorithm.

For primality *testing*, however, we can work in polynomial time. Some of these algorithms have been known since the 1970s.

Prime number theorem: The number of primes in $\{1,\dots,n\}\approx\frac{n}{\log(n)}$.

To gen an $n$-bit prime, pick a random $n$-bit value and check if prime. If not, repeat. We have the ability to efficiently generate large primes, and calculate $N$.

\end{document}
























