% -*-latex-*-
\documentclass[12pt]{article}

\usepackage{amssymb}
\usepackage{amsmath}
\usepackage{amsthm}
\usepackage{cancel}

\usepackage[paper=letterpaper,includehead,left=1in,right=1in,bottom=1in,top=.6in,headheight=.6in]{geometry}
\usepackage{fancyhdr}
\fancyhead[L]{University of Texas at Austin\\
Department of Computer Science}
\fancyhead[R]{Cryptography\\
CS 346, Spring 2016}
\pagestyle{fancy}

\frenchspacing

\newcommand{\AAA}{\mathcal{A}}
\newcommand{\CCC}{\mathcal{C}}
\newcommand{\FFF}{\mathcal{F}}
\newcommand{\KKK}{\mathcal{K}}
\newcommand{\MMM}{\mathcal{M}}

\newcommand{\Concat}{\parallel}
\newcommand{\eps}{\varepsilon}
\newcommand{\Enc}{\mathsf{Enc}}
\newcommand{\Dec}{\mathsf{Dec}}
\newcommand{\Mac}{\mathsf{Mac}}
\newcommand{\Macf}{\mathsf{Mac\text{-}forge}}
\newcommand{\Encf}{\mathsf{Enc\text{-}forge}}
\newcommand{\Vrfy}{\mathsf{Vrfy}}
\newcommand{\Decrypt}[2]{\Dec_{#1}(#2)}
\newcommand{\Encrypt}[2]{\Enc_{#1}(#2)}
\newcommand{\Gen}{\mathsf{Gen}}
\newcommand{\ang}[1]{\langle#1\rangle}
\newcommand{\GenEncDec}{(\Gen,\Enc,\Dec)}
\newcommand{\GenMacVrfy}{(\Gen,\Mac,\Vrfy)}
\newcommand{\ExptEavArgs}[2]{\mathsf{PrivK}^{\mathsf{eav}}_{#1,#2}}
\newcommand{\ExptCcaArgs}[2]{\mathsf{PrivK}^{\mathsf{CCA}}_{#1,#2}}
\newcommand{\ExptEav}{\ExptEavArgs{\AAA}{\Pi}}
\newcommand{\ExptCca}{\ExptCcaArgs{\AAA}{\Pi}}
\newcommand{\ExptPrgArgs}[2]{\mathsf{PRG}_{#1,#2}}
\newcommand{\ExptPrg}{\ExptPrgArgs{\AAA}{G}}
\newcommand{\Fcns}[1]{\mathsf{Func}_n}
\newcommand{\LengthKey}[1]{\ell_{\mathit{key}}(#1)}
\newcommand{\LengthInput}[1]{\ell_{\mathit{in}}(#1)}
\newcommand{\LengthOutput}[1]{\ell_{\mathit{out}}(#1)}
\newcommand{\xor}{\oplus}
\newcommand{\Pit}{\widetilde{\Pi}}
\newcommand{\negl}{{\tt negl}}

\newcommand{\Exec}[5]{\mathit{exec}^{#1}_{#2}(#3,#4,#5)}

\begin{document}

\title{CS 346 Class Notes}
\date{Feb 10, 2016}
\author{Mark Lindberg}
\maketitle
\thispagestyle{fancy}

{\bf Last Time:}

PRF $F$ gives a (strongly) secure $\Mac$ for $n$-bit messages. $\Pi'(\Mac',\Vrfy')$.

We can use the $\Mac$ above to produce a (strongly) secure $\Mac$ for arbitrary-length messages. $\Pi(\Mac,\Vrfy)$.

The construction of $\Pi$ was: $\Mac$: Divide the message $m$ into $\frac{n}{4}$-bit blocks. We have an $n$-bit key $k$. $m=m_1,m_2,\dots,m_d$ when broken up. $m_d$, the last block, is padded with $0$s if necessary. Let $\ell$ be the length of the message before padding. Then $t_i=\Mac'_k(r||\ang{\ell}||\ang{i}||m_i)$, where $r$ is an uniformly chosen $\frac{n}{4}$-bit string, which is used for the entire message $m$, then sent with for verification. Tag $=(r,t_1,t_2,\dots,t_d)$.

$\Vrfy(m,t)$, where $t$ is of the form above. Construct each of the blocks as above, using the given $r$ and $m$, and then run $\Mac'$ on each of them and check equality.

{\bf This Time:}

Theorem: This scheme is secure as long as $\Pi'$ is secure.

Recall: Our ``experiment'' for security is the $\Macf$ experiment. $\Pr[\Macf_{\AAA,\Pi}(n)=1]$ is $\negl$.

$\Macf$ means $\AAA$ gets a $\Mac_k$ oracle, $m_1\to t_1$, $m_2\to t_2$, etc., leading to an output of $(m,t)$. $A$ wins if for some ``new'' $m$ it chooses $\Vrfy_k(m,t)$ passes.

Fix an arbitrary PPT $A$.

Proof: We have $3$ events.\begin{itemize}

\item[$E_1$:] $\AAA$ succeeds.

\item[$E_2$:] ``Repeat'': Some $r$ repeats.

\item[$E_3$:] ``New Block'': Some $(r||\ang{\ell}||\ang{i}||m_i)$ is passed to $\Mac'_k$ when checking $\AAA$'s output is ``new''.

\end{itemize}

\begin{align*}
\Pr[E_1]&=\Pr[E_1\wedge E_2]+\Pr[E_1\wedge\overline{E_2}\wedge E_3]+\Pr[E_1\wedge\overline{E_2}\wedge\overline{E_3}]\\
&\leq\Pr[E_2]+\Pr[E_1\wedge E_3]+\Pr[E_1\wedge\overline{E_2}\wedge\overline{E_3}]
\end{align*}

Then $\Pr[E_2]=\mathcal{O}\left(\frac{q(n)}{2^n}\right)$ (I can't read that part of the board very well, so don't trust this.)

Also, $\Pr[[E_1\wedge\overline{E_2}\wedge\overline{E_3}]=0$, because $\overline{E_2}\wedge\overline{E_3}$ means we {\it have} to send an $m$ we've already seen, thus breaking success. That is, $\overline{E_2}\wedge\overline{E_3}\Rightarrow\overline{E_1}$.

Finally, $\Pr[E_1\wedge E_3]$. If this is not $\negl$, then $\Pi'$ is not secure. Assume we have an adversary $\AAA'$ for $\Pi'$. Then when $\AAA$ calls $\Mac_k(m)$, $\AAA'$ chooses a random $r$, forms blocks of the proper form, then uses oracle for $\Mac_k'$ to get $t_i'$s. When $\AAA$ outputs $(m,t)$, $\AAA'$ outputs...? It forms all $d$ blocks, and does something. Need to reference the book. Conclusion: $\negl(n)\geq\Pr[\AAA'{\text{ succeeds}}]\geq\Pr[E_1\cap E_3]$.

Therefore, $\Pr[E_1]\leq\negl(n)$.

------------------------------------------------------------------------------------------------------------------

Now we are going to examine a {\it practical} secure $\Mac$ for arbitrary length message. This is the CBC-MAC, Cipher Block Chaining Message Authentication Code.

Let $IV=0^n$. Run CBC, and only output $c_d$, the last block of the cipher-text produced by the CBC. We assume that all messages $m$ are of length $\ell(n)$-bits.

This doesn't work for arbitrary-length messages, so we can set some long length, and pad shorter messages. Or we could include some $F_k'$, and we output $F_k'(c_d)$. Either fix the arbitrary-length message.

\underline{Authenticated Encryption}:\begin{itemize}

\item Combines confidentiality and integrity.

\end{itemize}

An authenticated encryption scheme must be\begin{enumerate}

\item Unforgeable.

\item CCA-secure.

\end{enumerate}

What does it mean to be unforgeable? Let $\Pi=\GenEncDec$. $\Pi$ is said to be unforgeable if it passes the $\Encf_{\AAA,\Pi}(n)$ experiment.

The experiment:

$\AAA$ gets access to $\Enc_k$ oracle, and $\AAA$ must produce a ciphertext $c$. $\AAA$ wins the game if $\Dec_k(c)\neq\perp$. (?) And it is ``new'' (Not the output of some oracle query).

``Encrypt-then-authenticate''. Ingredients\begin{enumerate}

\item CPA-secure encryption scheme $\Pi_E$.

\item Strongly secure $\Mac$ $\Pi_M$.

\end{enumerate}

With different keys chosen uniformly at random.

\end{document}
























