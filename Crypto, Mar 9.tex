% -*-latex-*-
\documentclass[12pt]{article}

\usepackage{amssymb}
\usepackage{amsmath}
\usepackage{amsthm}
\usepackage{cancel}

\usepackage[paper=letterpaper,includehead,left=1in,right=1in,bottom=1in,top=.6in,headheight=.6in]{geometry}
\usepackage{fancyhdr}
\fancyhead[L]{University of Texas at Austin\\
Department of Computer Science}
\fancyhead[R]{Cryptography\\
CS 346, Spring 2016}
\pagestyle{fancy}

\frenchspacing

\newcommand{\Z}{\mathbb{Z}}
\newcommand{\AAA}{\mathcal{A}}
\newcommand{\CCC}{\mathcal{C}}
\newcommand{\FFF}{\mathcal{F}}
\newcommand{\KKK}{\mathcal{K}}
\newcommand{\MMM}{\mathcal{M}}
\newcommand{\TTT}{\mathcal{T}}

\newcommand{\Concat}{\parallel}
\newcommand{\eps}{\varepsilon}
\newcommand{\Enc}{\mathsf{Enc}}
\newcommand{\Dec}{\mathsf{Dec}}
\newcommand{\Mac}{\mathsf{Mac}}
\newcommand{\Macf}{\mathsf{Mac\text{-}forge}}
\newcommand{\Macsf}{\mathsf{Mac\text{-}sforge}}
\newcommand{\Encf}{\mathsf{Enc\text{-}forge}}
\newcommand{\Vrfy}{\mathsf{Vrfy}}
\newcommand{\Decrypt}[2]{\Dec_{#1}(#2)}
\newcommand{\Encrypt}[2]{\Enc_{#1}(#2)}
\newcommand{\Gen}{\mathsf{Gen}}
\newcommand{\ang}[1]{\langle#1\rangle}
\newcommand{\GenEncDec}{(\Gen,\Enc,\Dec)}
\newcommand{\GenMacVrfy}{(\Gen,\Mac,\Vrfy)}
\newcommand{\ExptEavArgs}[2]{\mathsf{PrivK}^{\mathsf{eav}}_{#1,#2}}
\newcommand{\ExptCcaArgs}[2]{\mathsf{PrivK}^{\mathsf{CCA}}_{#1,#2}}
\newcommand{\ExptCpaArgs}[2]{\mathsf{PrivK}^{\mathsf{CPA}}_{#1,#2}}
\newcommand{\ExptHCArgs}[2]{\mathsf{Hash\text{-}coll}_{#1,#2}}
\newcommand{\ExptEav}{\ExptEavArgs{\AAA}{\Pi}}
\newcommand{\ExptCca}{\ExptCcaArgs{\AAA}{\Pi}}
\newcommand{\ExptCpa}{\ExptCpaArgs{\AAA}{\Pi}}
\newcommand{\ExptHC}{\ExptHCArgs{\AAA}{\Pi}}
\newcommand{\ExptPrgArgs}[2]{\mathsf{PRG}_{#1,#2}}
\newcommand{\ExptPrg}{\ExptPrgArgs{\AAA}{G}}
\newcommand{\Fcns}[1]{\mathsf{Func}_n}
\newcommand{\LengthKey}[1]{\ell_{\mathit{key}}(#1)}
\newcommand{\LengthInput}[1]{\ell_{\mathit{in}}(#1)}
\newcommand{\LengthOutput}[1]{\ell_{\mathit{out}}(#1)}
\newcommand{\xor}{\oplus}
\newcommand{\Pit}{\widetilde{\Pi}}
\newcommand{\negl}{{\tt negl}}

\newcommand{\Exec}[5]{\mathit{exec}^{#1}_{#2}(#3,#4,#5)}

\begin{document}

\title{CS 346 Class Notes}
\date{Mar 9, 2016}
\author{Mark Lindberg}
\maketitle
\thispagestyle{fancy}

{\bf Last Time:}

Apparently $\epsilon$ is the empty string.

{\bf This Time:}

To find the loop in a linked list (useful for finding collisions in hash function), maintain 2 pointers, $A$ and $B$. Each step, move $A$ forward 1 step, and $B$ forward 2 steps. Repeat this process until the pointers hold the same value. At time $t$, then $B$ has advance $2t$ steps, and $A$ has advance $t$ steps. When they collide, they must both be somewhere in the cycle itself.

Because $B$ has advanced $t$ more steps than $A$, then $t$ must be a multiple of $s$, the cycle length. The first multiple of $s\geq r$, where $r$ is the number of steps from the beginning of the list until the beginning of the cycle is reached.

Let $k=\left\lfloor\frac{r}{s}\right\rfloor$, and $\ell=r\mod{s}$. Then $r=ks+\ell$.

If $\ell=0$, then $r$ is a multiple of $s$, and $t=ks$.

Else, $\ell>0$, then they meet at $t=(k+1)s$. Then the loop index of the meeting point is $(k+1)s-(ks+\ell)=s-\ell$.

Now run pointers $C$ and $D$ at speed 1, one from the head of the list, one from the collision point.

$C$ reaches the loop at step $i\cdot s+\ell$ for $i\geq k$.

$D$ reaches loop index $0$ at step $i\cdot s+\ell$ for $i\geq 0$.

They will meet at step $ks+\ell$.

\underline{Random Oracle Model}:

The Algorithm has access to an oracle for a random function from $\ell_{\in}(n)$-bits to $\ell_{out}(n)$-bits. Compare this to $f$ in the definition of a PRF.

We can easily construct a PRG, collision resistant hash function, or a PRF in the random-oracle model.\begin{enumerate}

\item $\ell_{in}(n)<\ell_{out}(n)$, get a PRG such that $\left|\Pr[D^H(y)=1]-\Pr[D^H(H(x))=1]\right|\leq\negl(n)$.

\item $\ell_{in}>\ell_{out}(n)$, $\Omega\left(2^{\ell_{out}(n)/2}\right)$.

\item $\ell_{in}(n)=2n$. $\ell_{out}(n)=n$. $F_k(x)=F(k,x)=y$, $F_k(x)=H(k||x)$.

\end{enumerate}

Additional applications of hash functions:\begin{enumerate}

\item We can track hash values of known viruses.

\item Deduplication in cloud storage.

\item File location/load balancing in P2P systems.

\item Merkle trees.

\end{enumerate}

\underline{Password Hashing!}\begin{itemize}

\item Store a hash of each user's PWD in the PWD file.

\item This means that is it sufficient for the attacker to find \underline{\it a} preimage of your PWD, which may or may not be your password.

\end{itemize}

The second item may be weak to something called Hellman's scheme.

Preprocessing time $2^{\ell}$. Uses $2^{\frac{2}{3}\ell}$ space.

Subsequently, can answer a preimage query in $2^{\frac{2}{3}\ell}$ time.

Mitigation techniques to fight these attacks:\begin{enumerate}

\item Use a slow hash function--make it take maybe half a second.

\item Add a salt to every password.

\end{enumerate}

\end{document}
























