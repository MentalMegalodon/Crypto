% -*-latex-*-
\documentclass[12pt]{article}

\usepackage{amssymb}
\usepackage{amsmath}
\usepackage{amsthm}
\usepackage{cancel}

\usepackage[paper=letterpaper,includehead,left=1in,right=1in,bottom=1in,top=.6in,headheight=.6in]{geometry}
\usepackage{fancyhdr}
\fancyhead[L]{University of Texas at Austin\\
Department of Computer Science}
\fancyhead[R]{Cryptography\\
CS 346, Spring 2016}
\pagestyle{fancy}

\frenchspacing

\newcommand{\AAA}{\mathcal{A}}
\newcommand{\CCC}{\mathcal{C}}
\newcommand{\FFF}{\mathcal{F}}
\newcommand{\KKK}{\mathcal{K}}
\newcommand{\MMM}{\mathcal{M}}

\newcommand{\Concat}{\parallel}
\newcommand{\eps}{\varepsilon}
\newcommand{\Enc}{\mathsf{Enc}}
\newcommand{\Dec}{\mathsf{Dec}}
\newcommand{\Mac}{\mathsf{Mac}}
\newcommand{\Macf}{\mathsf{Mac\text{-}forge}}
\newcommand{\Macsf}{\mathsf{Mac\text{-}sforge}}
\newcommand{\Encf}{\mathsf{Enc\text{-}forge}}
\newcommand{\Vrfy}{\mathsf{Vrfy}}
\newcommand{\Decrypt}[2]{\Dec_{#1}(#2)}
\newcommand{\Encrypt}[2]{\Enc_{#1}(#2)}
\newcommand{\Gen}{\mathsf{Gen}}
\newcommand{\ang}[1]{\langle#1\rangle}
\newcommand{\GenEncDec}{(\Gen,\Enc,\Dec)}
\newcommand{\GenMacVrfy}{(\Gen,\Mac,\Vrfy)}
\newcommand{\ExptEavArgs}[2]{\mathsf{PrivK}^{\mathsf{eav}}_{#1,#2}}
\newcommand{\ExptCcaArgs}[2]{\mathsf{PrivK}^{\mathsf{CCA}}_{#1,#2}}
\newcommand{\ExptCpaArgs}[2]{\mathsf{PrivK}^{\mathsf{CPA}}_{#1,#2}}
\newcommand{\ExptEav}{\ExptEavArgs{\AAA}{\Pi}}
\newcommand{\ExptCca}{\ExptCcaArgs{\AAA}{\Pi}}
\newcommand{\ExptCpa}{\ExptCpaArgs{\AAA}{\Pi}}
\newcommand{\ExptPrgArgs}[2]{\mathsf{PRG}_{#1,#2}}
\newcommand{\ExptPrg}{\ExptPrgArgs{\AAA}{G}}
\newcommand{\Fcns}[1]{\mathsf{Func}_n}
\newcommand{\LengthKey}[1]{\ell_{\mathit{key}}(#1)}
\newcommand{\LengthInput}[1]{\ell_{\mathit{in}}(#1)}
\newcommand{\LengthOutput}[1]{\ell_{\mathit{out}}(#1)}
\newcommand{\xor}{\oplus}
\newcommand{\Pit}{\widetilde{\Pi}}
\newcommand{\negl}{{\tt negl}}

\newcommand{\Exec}[5]{\mathit{exec}^{#1}_{#2}(#3,#4,#5)}

\begin{document}

\title{CS 346 Class Notes}
\date{Feb 22, 2016}
\author{Mark Lindberg}
\maketitle
\thispagestyle{fancy}

{\bf Last Time:}
\begin{tabular}{| l | l |}
\hline
Experiment & Security notion\\\hline
$\Macsf_{\AAA,\Pi}(n)$ & Strongly secure mac.\\\hline
$\ExptCpa(n)$ & CPA-secure encryption scheme.\\\hline
$\ExptCca(n)$ & CCA-secure encryption scheme.\\\hline
$\Encf_{\AAA,\Pi}(n)$ & Unforgeable encryption scheme.\\\hline
\end{tabular}

An authenticated encryption scheme should be\begin{enumerate}

\item Unforgeable.

\item CCA-secure.

\end{enumerate}

We'll get the above by combining a CPA-secure scheme with a strongly secure MAC.

{\bf This Time:}

In the $\Macsf_{\AAA,\Pi}(n)$ experiment the adversary $\AAA$ gets access to $\Mac_k$ oracle which outputs $(m,t)$ The adversary succeeds if it can output a pair $(m,t)$ which is new such that $\Vrfy_k(m,t)=1$.

In the $\ExptCpa(n)$ experiment, the challenger generates $k$. The adversary gets access to $\Enc_k$ oracle. The adversary picks $m_0,m_1$ of the same length, sends them to the challenger. The challenger picks a random bit $b$, encrypts $c=\Enc_k(b)$, and sends back $c$. The adversary then has more oracle access to $\Enc_k$. (Not on $m_0$ or $m_1$, though.)

In the $\ExptCca(n)$ experiment, much is the same as the previous experiment, except that the adversary also gets access to a $\Dec_k$ oracle, though they cannot call it on the $c$ produced by the challenger.

In the $\Encf_{\AAA,\Pi}(n)$ experiment, the adversary gets access to $\Enc_k$ oracle. The adversary outputs $c$. The adversary succeeds if\begin{enumerate}

\item $\Dec_k(c)=1$. ($c$ is a valid ciphertext.)

\item $\Dec_k(c)$ is not a message we queried the oracle on previously.

\end{enumerate}

``Encrypt-then-authenticate'' paradigm. Let $\Pi_E=(\Enc,\Dec)$ be a CPA-secure encryption scheme, and $\Pi_M=(\Mac,\Vrfy)$ be a strongly secure MAC. ($\Gen$ is dropped because we can use the same one for both, but don't want name collisions.) Then $\Pi:=(\Gen',\Enc',\Dec')$.

$\Gen'$ generates independent $n$-bit keys $k_E$ and $k_M$. Let $k=(k_E,k_M)$.

$\Enc'_k(m)=(c,\Mac_{k_M}(c))$ where $c=\Enc_{k_E}(m)$.

$\Dec'_k((c,t))$. First, verify that $\Vrfy_{k_M}(c,t)=1$. If not, return $\perp$. If so, then return $\Dec_{k_E}(c)$.

THEOREM: $\Pi'$ is an authentication scheme as defined above.

Intuitive proof outline: Consider an adversary $\AAA$ in the CCA experiment. We'll show that whp (with high probability), all calls to $\Dec'_k$ one outputs $\perp$ unless they correspond to a call to $\Enc'_k$. A call comes to $\Enc'_k$, which comes from the use of $\Pi_M$, which yields unforgeablity. Consequently, the ability to call $\Dec'_k$ is ``useless'', so the CPA security of $\Pi_E$ will be enough.

Some proof details:

``Valid query'' event. Call to $\Dec'_k$ with $(c,t)$ such that\begin{enumerate}

\item $(c,t)$ is not output of prior $\Enc'_k$ call.

\item $\Dec'_k(c,t)\neq\perp$.

\end{enumerate}

Claim: $\Pr[\text{valid query}]\leq\negl(n)$.

Assume $\AAA$ makes $\leq q(n)$ $\Dec$ queries, where $q$ is polynomial.

Simulation argument: Construct an adversary $\AAA_M$ from $\AAA$, in the $\Macsf_{\AAA_M,\Pi_M}(n)$ experiment. It has access to $\Mac_{k_M}$. At the beginning, $\AAA_M$ chooses a random $k_E$ when $\AAA$ call $\Enc'_k(m)$. $\AAA_M$ simulates this call\begin{enumerate}

\item can compute $c=\Enc_{k_E}(m)$.

\item uses oracle to get $t$.

\end{enumerate}

If $(c,t)$ is the output of a previous call to $\Enc'_k(m)$, return $m$. Else, if this is the $i$th nontrivial query, where $i$ is a random number we picked, halt and output $(c,t)$. Else, return $\perp$.

$$\Pr\left[\Macsf_{\AAA_M,\Pi_M}(m)=1\right]\geq\frac{\Pr[\text{valid query}]}{q(n)}$$

This is definitely the most complicated/confusing proof we've seen so far...

\end{document}
























